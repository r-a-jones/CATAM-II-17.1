\documentclass{article}
\usepackage[utf8]{inputenc}
\usepackage{amssymb}
\usepackage{amsmath}
\usepackage{hyperref}
\usepackage{pifont}
\usepackage[table]{xcolor}
\usepackage{centernot}
\usepackage{mathtools}
\usepackage{tabularx}
\usepackage{imakeidx}
\usepackage{booktabs}
\usepackage{pgfplots}
\usepackage{pgfplotstable}
\usepackage{adjustbox}
\usepackage{tkz-euclide}
\usepackage{dashbox}
\usepackage{listings}
\usepackage{verbatim}
\usepackage{tcolorbox}
\usepackage{enumerate}
\usepackage{pmboxdraw}
\usepackage{graphicx}
\usepackage{adjustbox}
\usepackage{caption}
\usepackage{wrapfig}
\usepackage{subcaption}
\usepackage{listings}
\usepackage{scrextend}
\usepackage{float}
\usepackage{caption}
\usepackage{makecell}
\usepackage{footnote}
\usepackage{enumitem}
\usepackage{amsthm}
\usepackage{filecontents}
\usepackage{graphicx,nicefrac}

\pgfplotstableset{
	color cells/.style={
		col sep=comma,
		string type,
		postproc cell content/.code={%
			\pgfkeysalso{@cell content=\rule{0cm}{2.4ex}\cellcolor{black!##1}\pgfmathtruncatemacro\number{##1}\ifnum\number>50\color{white}\fi##1}%
		},
		columns/x/.style={
			column name={},
			postproc cell content/.code={}
		}
	}
}

\makesavenoteenv{tabular}
\newfloat{Algorithm}{htbp}{loa}

\newcommand{\cmark}{\checkmark}%

\newcommand{\xmark}{\ding{55}}%

\newcommand{\legendre}[2]{\left( \frac{#1}{#2} \right)}

\lstset{
	basicstyle=\small\ttfamily,
	columns=flexible,
	breaklines=true
}
\usepackage[a4paper, total={6in, 9in}]{geometry}

\setlength{\parindent}{0pt}

\title{17.1 Graph Colouring}
\date{Easter 2024}
\setlength{\parskip}{0.25em}

\setcounter{secnumdepth}{0}

%\newcommand{\projectquote}[1]{\noindent {\fbox{\begin{minipage}{45em}
				%#1
				%\end{minipage}}}
				%\vspace{3mm}
				%}
			\newcommand{\projectquote}[1]{\noindent {\fbox{\begin{minipage}{45em}
							#1
				\end{minipage}}}
				\vspace{3mm}
			}
			
			\newcommand{\algorithmquote}[1]{\noindent
				{\fbox{\begin{minipage}{44em}
							#1
				\end{minipage}}}
			}
			
			\newcommand{\programquote}[1]{\fontfamily{pcr}\selectfont 
				#1\normalfont
			}
			
			\newcommand{\namepageref}[1]{``\nameref{#1}'' (p. \pageref{#1})}
			
			
			
			
			\definecolor{dkgreen}{rgb}{0,0.6,0}
			\definecolor{gray}{rgb}{0.5,0.5,0.5}
			\definecolor{mauve}{rgb}{0.58,0,0.82}
			
			\begin{document}
				
				\maketitle
				\tableofcontents
				

\newtheorem{lemma}{Lemma}


\newcommand\bsfrac[2]{%
	\scalebox{-1}[1]{\nicefrac{\scalebox{-1}[1]{$#1$}}{\scalebox{-1}[1]{$#2$}}}%
}


\section{17.1 Graph Colouring}
This project's code has been written in C\#. The code can be found in the appendix at \namepageref{appendix_a_code}. The output can be found in \namepageref{appendix_b_output}.

Throughout this project the notation $K_n$ will be used to refer to a complete graph on $n$ vertices.

The \emph{clique number} of a graph $G$ is the largest $n$ such that there exists a $K_{n} \subseteq G$.
		
\subsection{Question 1}

The output can be found in \namepageref{output_question_1}.
				
The methods i-iv as discussed in the project instructions have been implemented, as well as the greedy algorithm. The following tables show the number of colours used by the greedy algorithm for each randomly generated graph in the $\mathcal{G}(70,0.5)$ and $\mathcal{G}_3(70,0.5)$ cases respectively.

\begin{center}
	\pgfplotstabletypeset[
	columns={graphNum, ascending, descending, subgraphs, random},
	columns/graphNum/.style={column name={Graph tested}, string type},
	columns/ascending/.style={column name={Method (i)}, precision=5},
	columns/descending/.style={column name={Method (ii)}, precision=5},
	columns/subgraphs/.style={column name={Method (iii)}, precision=5},
	columns/random/.style={column name={Method (iv)}, precision=5},
	every column/.style={column type/.add={|}{}},
	every last column/.style={column type/.add={}{|}},
	every head row/.style={before row=\hline, after 	row=\hline},
	every last row/.style={after row=\hline},
	]{Program Output/Question1_Table_05.txt}
	\captionof{table}{Number of colours used to colour ten randomly generated graphs in $\mathcal{G}(70,0.5)$.}
\end{center}

\begin{center}
	\pgfplotstabletypeset[
	columns={graphNum, ascending, descending, subgraphs, random},
	columns/graphNum/.style={column name={Graph tested}, string type},
	columns/ascending/.style={column name={Method (i)}, precision=5},
	columns/descending/.style={column name={Method (ii)}, precision=5},
	columns/subgraphs/.style={column name={Method (iii)}, precision=5},
	columns/random/.style={column name={Method (iv)}, precision=5},
	every column/.style={column type/.add={|}{}},
	every last column/.style={column type/.add={}{|}},
	every head row/.style={before row=\hline, after 	row=\hline},
	every last row/.style={after row=\hline},
	]{Program Output/Question1_Table_075.txt}
	\captionof{table}{Number of colours used to colour ten randomly generated graphs in $\mathcal{G}_3(70,0.75)$.}
\end{center}

All the details of the generated graphs (i.e. the precise structure of them) can be found in the program output in the appendix.


As can be seen in both cases, each of the four methods for labelling the vertices are about as equally effective as each other when using the greedy algorithm. % In particular, methods (i), (ii) and (iv) produce incredibly similar averages in both cases, with (iii) slightly worse in the first case, and slightly better in the second case in our given sample. (Taking a larger sample suggests this is an outlier and that the four methods are indeed about the same on average.)

Graphs randomly selected from $\mathcal{G}_3(70,0.75)$ notably require much fewer colours on average than those in $\mathcal{G}(70,0.5)$ via the greedy algorithm. This is notable, as the expected number of edges of a graph in $\mathcal{G}(70,0.5)$ is

$$\binom{70}{2} \cdot 0.5 = 1207.5,$$

which is lower than the expected number of edges of a graph in $\mathcal{G}_3(70,0.75)$ which is

$$\left(\binom{70}{2} - \underbrace{\binom{24}{2} - 2 \cdot \binom{23}{2}}_{(\ast)}\right) \cdot 0.75 = 1224.75,$$

[$(\ast)$ accounts for the pairs of vertices $i,j$ with $i-j \equiv 0 \mod 3$, which shouldn't be counted.]

despite the much higher average number of colours used in the case of $\mathcal{G}(70,0.5)$.

This result does however make sense when we note that the rule that no edge $ij$ can exist with $i-j \equiv 0 \mod 3$ gives rise to a lot of structure within a graph randomly chosen from $\mathcal{G}_3(70,0.75)$, allowing a colouring with few colours to arise more easily, whereas a graph randomly chosen from $\mathcal{G}(70,0.5)$ is expected to have no such structure, meaning little symmetry can be exploited by a colouring to allow for a small number of colours to exist.


%As can be seen in both cases, method (iii), that is, the ordering where $v_j$ has minimum degree in the graph $G - \{v_{j+1}, . . . , v_n\}$, is much more effective than the other three methods, which are more-or-less equally as effective as each other in each case.

%We may explain why (iii) is more effective in the second case by noting that its method of ordering the vertices compliments how the greedy algorithm functions. Since the greedy algorithm will tend to colour a vertex with a larger number in cases where the vertex has a large number of neighbours proceeding it in the ordering, this method of labelling the vertices in such a way as to minimise the number of neighbours with an earlier position in the ordering will reduce the likelihood of such a case occurring. 


\subsection{Question 2}

\subsubsection{An ordering guaranteeing that the greedy algorithm uses no more than 3 colours for $\mathcal{G}_3(70, 0.75)$}

An order guaranteeing such a result can be obtained as follows:

Suppose $v_1, \dots, v_{70}$ is the original labelling of the graph's vertices after generating it from $\mathcal{G}_3(70, 0.75)$. In particular, this labelling obeys the rule that $v_i v_j$ is not an edge when $i - j \equiv 0 \mod 3$.

We construct a new labelling $v'_1, \dots, v'_{70}$ as follows, where the square brackets denote ordered sets:

$$[v'_1, \dots, v'_{70}] = [v_1, v_4, v_7, \dots, v_{67}, v_{70} v_2, v_5, v_8, \dots, v_{68}, v_3, v_6, v_9, \dots, v_{69}].$$

That is, we group the original $v_i$s by congruency class: require all $v_i$ with $i \equiv 1 \mod 3$ to come first, followed by the $v_i$ with $i \equiv 2 \mod 3$, then the $v_i$ with $i \equiv 0 \mod 3$.

Such an ordering guarantees that the vertices that were originally $v_i$ with $i \equiv 1 \mod 3$ will be labelled with the colour $1$, due to no edges existing between these vertices. The $i \equiv 2 \mod 3$ will then be labelled with at most the colour $2$ similarly (some may be coloured with $1$) and the final congruence class labelled with at most the colour $3$.

This idea can be easily extended to give an ordering that guarantees a graph selected from $\mathcal{G}_k(n, p)$ will be coloured in no more than $k$ colours.


\subsubsection{Why was $p=0.75$ chosen here?}

As per our discussion in question 1, we note that $$\mathbb{E}(\text{\#edges of $G$ chosen from $\mathcal{G}(70, 0.5)$}) \approx \mathbb{E}(\text{\#edges of $G$ chosen from $\mathcal{G}_3(70, p)$})$$ when $p=0.75$. Despite the similar number of edges, in question 1 we remarked that the greedy algorithm is able to find a much better colouring $\mathcal{G}_3(70, 0.75)$ due to the structure that arises in such graphs. In this question we proved that such a graph can indeed by coloured by greedy in at most $3$ colours by exploiting this structure.

No such structure can be exploited for a graph $G$ chosen in $\mathcal{G}(70, 0.5)$, meaning we are unable to do similar analysis to give an upper bound on $\chi(G)$, let alone provide an ordering such that the greedy algorithm constructs it. Indeed, it is hypothetically possible we pick $G = K_{70}$, the complete graph on $70$ vertices, requiring $70$ colours.

\subsubsection{A graph $G$ of order $3n$ such that $\chi(G) = 3$ but on which greedy might need $n + 2$ colours}

Consider the graph $G$ as defined. $G$ has vertices labelled $\{ v_{i,j} \mid 1 \leq i \leq 3, 1 \leq j \leq n\}$, in particular $3n$ vertices. Say $v_{i,j}$ is connected to $v_{i',j'}$ provided either:

\begin{itemize}
	\item $i \neq i'$ and $j \neq j'$,
	\item $j = j' = n$.
\end{itemize}

The following demonstrates a $3$-colouring of $G$:

Let $A_i = \{v_{i, j} \mid 1 \leq j \leq n \}$ for $i = 1, 2, 3$. Then colour all the vertices in the set $A_i$ with the colour $i$, $i=1,2,3$. This is a $3$-colouring.

The greedy algorithm produces an $n+2$-colouring if we order the vertices as $v'_1, \dots, v'_{3n}$ as follows:

$$[v'_1, \dots, v'_{3n}] = [v_{1,1}, v_{2,1}, v_{3,1}, v_{1,2}, v_{2,2}, v_{3,2}, \dots v_{1,n}, v_{2,n}, v_{3,n}].$$

Under this ordering: $v_{1,1}, v_{2,1}, v_{3,1}$ are not connected to each other, so they all get coloured $1$. $v_{1,2}, v_{2,2}, v_{3,2}$ are all not connected to each other, but are connected to two of $v_{1,1}, v_{2,1}, v_{3,1}$, thus get coloured $2$. $v_{1,3}, v_{2,3}, v_{3,3}$ are not connected to each other but are connected to two of those vertices we coloured $1$ and two of those vertices we coloured $2$. Thus they get coloured $3$. In general $v_{1,j}, v_{2,j}, v_{3,j}$ get coloured $j$ for $j<n$. Then $v_{1,n}, v_{2,n}, v_{3,n}$ form a triangle, with each connected to something of colour $1, 2, \dots, n-1$, thus they get coloured $n, n+1, n+2$ respectively.

A similar construction can be used to find a graph $G$ of order $kn$ such that $\chi(G) = k$, but which greedy uses $n+k-1$ colours.

\subsection{Question 3}

\subsubsection{An argument to suggest that it is unlikely the greedy-type algorithm will find a complete subgraph of order $14$ in a graph from $\mathcal{G}(2000, 0.5)$.}

First, note that the probability that any given $14$ randomly selected vertices selected from such a graph forms a $K_{14}$ is $\left( \frac{1}{2} \right)^{\binom{14}{2}} \approx 4.04 \cdot 10^{-28}$.

The way this greedy algorithm works, given a sequence $v_1, \dots, v_{14}$ of vertices in the graph, we first check if $v_2$ is connected to $v_1$, requiring $1$ operation to test, with a probability of succeeding being $\frac{1}{2}$. If we fail, we terminate this process. We then check if $v_3$ is connected to $v_1$ and $v_2$, requiring $2$ operation to test, with a probability of succeeding being $\frac{1}{4}$, if we fail we terminate, and so on.

On average, the number of operations that this algorithm uses per set of input vertices $v_1, \dots, v_{14}$ is

$$\begin{aligned}
	\mathbb{E}(\text{operations}) &= 1 + \frac{1}{2}(2 + \frac{1}{4}(3 + \cdots)) \\
	&= 1 + \frac{1}{2} \cdot 2 + \frac{1}{2^3} \cdot 3 + \dots + \frac{1}{2^{\binom{14}{2}}} \cdot 14 \\
	&\approx 2.44.
\end{aligned}$$

but, with a success chance of $\left( \frac{1}{2} \right)^{\binom{14}{2}}$.

Thus, if we were to pick $14$-tuples of vertices at random from the graph until we successfully found a $K_{14}$, it will take about $2^{\binom{14}{2}} \cdot 2.44 \approx 6.05 \cdot 10^{27}$ operations to do so, i.e. the program will not terminate in a reasonable amount of time.

We can optimise this algorithm somewhat by noting that if we have already found, say, a $K_{13}$ within the vertices $v_1, \dots, v_{13}$, it is worth testing all tuples $(v_1, \dots, v_{13}, v)$ for $v$ a vertex in the graph distinct from $v_1, \dots, v_{13}$ for a $K_{14}$. Similarly, if we have found a  $K_{12}$ we try using it to construct a $K_{13}$, and so on (i.e. the algorithm traverses a tree-like structure representing all possible pairs, triplets, ..., $14$-tuples of vertices of the graph, in a depth-first manner.) This optimisation will however still result in an incredibly large number of operations before a $K_{14}$ is likely to be found, due to the sheer size of the graph, and the exponentials involved in the probabilities we may increase our clique size at any given stage of the algorithm.

Note that while it is incredibly \emph{unlikely} that such an algorithm will find a $K_{14}$ in a graph randomly chosen from $\mathcal{G}(2000, 0.5)$, it is incredibly \emph{likely} that such a graph contains a $K_{14}$. As we argue in the next part of this question, the clique number of such a graph is almost surely between 16 and 18, 

\subsubsection{How large do you think a clique is likely to be in a graph from $\mathcal{G}(2000, 0.5)$?}



The following is a theorem proven in II Graph Theory\footnote{The lecturer for this year, 2023-4, provides it in his notes as "Theorem 5" of "Chapter 6: Random Graphs", located on page 39 (page 41 in the linked PDF document). A sketch proof is also provided there. See \url{https://tartarus.org/gareth/maths/notes/ii/Graph_Theory.pdf}.}, which we can apply to this question:

\emph{Theorem.} Let $G$ be chosen from $\mathcal{G}(n,p)$. Let $d \in \mathbb{R}_{>0}$ be the positive number such that $\binom{n}{d} p^{\binom{d}{2}} = 1$ (where we extend the binomial coefficient $\binom{\ast}{\ast}$ to the real numbers via the gamma function.) Then, almost surely, $G$ has clique number equal to $\lceil d \rceil$, $\lfloor d \rfloor$ or $\lfloor d \rfloor - 1$.

Applying this theorem, for $n=2000$, $p=0.5$, we find $d \approx 17.2$. Thus, almost surely, $G$ has its largest clique of size $16, 17$ or $ 18$. As remarked in the previous section of this question, this means that almost surely there is a $K_{14} \subseteq G$, despite the fact that the greedy-like algorithm is incredibly unlikely to find a $K_{14}$.

\subsection{Question 4}

\subsubsection{How the clique-finding algorithm works}

Say the graph $G$ has $n$ vertices, ordered as $v_1, \dots, v_n$. We find all the complete subgraphs of the graph $G$ recursively. Call $\mathcal{K}_i$ the set of all $i$-tuples of vertices of $G$ that form a $K_i$.

\begin{itemize}
	\item $\mathcal{K}_1$ is just the set of all vertex singletons in $G$. We initialise it in the program by just looping through $j=1, \dots, n$ and adding $\{ v_j\}$ to it.
	
	\item From $\mathcal{K}_i$ we can deduce $\mathcal{K}_{i+1}$:
	
	\subitem For each $K \in \mathcal{K}_i$:
	
	\subsubitem Let $m$ be the maximal $l$ such that $v_l \in K$ (e.g $m=4$ for $K = \{v_1, v_3, v_4\}$). For each $j = m+1, \dots, n$, test to see if $K \cup \{ v_j\}$ is a complete subgraph (by testing if $v_l v_j$ is an edge for all $v_l \in K$). If so, add $K \cup \{ v_j\}$ to $\mathcal{K}_{i+1}$.
	
	\subitem Is $\mathcal{K}_{i+1}$ empty? If so, $i$ is the clique number of $G$, and the elements of $\mathcal{K}_{i}$ are the cliques. Return this result.
	
	\subitem If not, then increment $i$.
	
	\subsubitem This process will surely terminate because we will find $\mathcal{K}_{n+1} = \emptyset$ (no complete subgraph with $n+1$ vertices exists because only $n$ vertices exist.)
\end{itemize}

The use of $m$ as defined "Let $m$ be the maximal $l$ such that..." in the program ensures efficiency and a lack of duplicate cliques: if, say, $\{ v_1, v_2, v_3\}$ were a clique, we would find that clique via $\{ v_1, v_2\}$ in the previous step which was in turn found via $\{ v_1\}$, but it would not be found from any other $I \subsetneq \{ v_1, v_2, v_3\}$ as a previous step.

\subsubsection{A human-verifiable example of the clique algorithm working}

I have generated a graph with $10$ vertices in order to test this clique-finding algorithm. The output here is more human-friendly and can be verified much faster, the program claims to find two distinct cliques of size 5 in this graph, it can be verified that this is indeed the case:

\begin{tcolorbox}[size=small]
	\begin{verbatim}
		Finding the cliques for the following graph:
		VertexOrderedGraph(10, (1, 2) (1, 3) (1, 4) (1, 5) (1, 6) (2, 3) (2, 4) (2, 5)
		(2, 6) (2, 7) (2, 8) (2, 9) (3, 4) (3, 7) (3, 8) (4, 5) (4, 6) (4, 8) (4, 9)
		(5, 6) (5, 8) (5, 10) (6, 8) (6, 9) (7, 8) (7, 9) (7, 10) (9, 10))
		It has clique number 5
		Clique(s) found:
		[1, 2, 4, 5, 6]
		[2, 4, 5, 6, 8]
	\end{verbatim}
\end{tcolorbox}

The line beginning \programquote{VertexOrderedGraph} tells us the information specifiying the graph - the initial \programquote{10} tells us it is a graph on $10$ vertices, and each \programquote{(x,y)} specifies that $xy$ is an edge of the graph. It is possible to check manually that the two cliques listed are in fact complete subgraphs of the graph, and no $K_6$ appears in this graph.

\subsubsection{Compare, for several graphs, the resulting lower bound you get on $\chi(G)$ with the upper bounds obtained previously.}

As before, we generate $10$ random graphs from $\mathcal{G}(70, 0.5)$ and $\mathcal{G}_3(70, 0.75)$ each. (Note different graphs as in question 1 have been generated.)

The greedy algorithm values for the orderings (i)-(iv) have been calculated as in question 1. Their minimum has been provided, giving an upper bound on $\chi(G)$. The clique numbers have been calculated giving a lower bound on $\chi(G)$.

The following values were found:

\begin{center}
	\pgfplotstabletypeset[
	columns={num, upperBound, lowerBound},
	columns/num/.style={column name={Graph tested}, string type},
	columns/upperBound/.style={column name={Upper bound}, precision=5},
	columns/lowerBound/.style={column name={Lower bound}, precision=5},
	every column/.style={column type/.add={|}{}},
	every last column/.style={column type/.add={}{|}},
	every head row/.style={before row=\hline, after 	row=\hline},
	every last row/.style={after row=\hline},
	]{Program Output/Question4_Table_05.txt}
	\captionof{table}{Upper and lower bounds on the chromatic number calculated by the question 1 \& clique algorithms for ten graphs in $\mathcal{G}(70,0.5)$.}
\end{center}

\begin{center}
	\pgfplotstabletypeset[
	columns={num, upperBound, lowerBound},
	columns/num/.style={column name={Graph tested}, string type},
	columns/upperBound/.style={column name={Upper bound}, precision=5},
	columns/lowerBound/.style={column name={Lower bound}, precision=5},
	every column/.style={column type/.add={|}{}},
	every last column/.style={column type/.add={}{|}},
	every head row/.style={before row=\hline, after 	row=\hline},
	every last row/.style={after row=\hline},
	]{Program Output/Question4_Table_075.txt}
	\captionof{table}{Upper and lower bounds on the chromatic number calculated by the question 1 \& clique algorithms for ten graphs in $\mathcal{G}_3(70,0.75)$.}
\end{center}

As before, all the details of the generated graphs can be found in the program output in the appendix.

As can be seen in the case of the graphs in $\mathcal{G}_3(70,0.75)$, the program was able to specify the chromatic number in nine of ten cases, as the upper bound is equal to the lower bound. Note that as per our analysis in question 2, we know $3$ is an upper bound for the chromatic number of any element of $\mathcal{G}_3(70,0.75)$, thus we can determine outside the program that all ten of the generated graphs have $\chi(G) = 3$.

Indeed, an element of $\mathcal{G}_3(70,0.75)$ is incredibly likely to have a chromatic number of exactly $3$ - we have proven $3$ is an upper bound, for $1$ to occur we must have precisely zero edges generate (there are $1633$ possible edges, thus the probability this occurs is $0.25^{1633}$, essentially zero) or no triangles generate (a more long-winded combinatorial argument can also show this probability is negligible).

Recall the theorem stated in question 3 regarding what the clique number of $G \in \mathcal{G}(n,p)$ almost surely is. For $n=70$, $p=0.5$, we can compute that this theorem tells us the clique number is almost surely $7, 8$ or $9$ (we get $d \approx 8.99$) which indeed matches our results in that table.

\subsection{Question 5}

\subsubsection{Calculating a maximal independent set via the clique algorithm}

For a graph $G$, say $G^c$ is the graph on the same vertex set as $G$ but with $ij$ an edge of $G \iff ij$ not an edge of $G^c$.

It follows that any complete subgraph of $G^c$ corresponds to an independent set of $G$, and thus that a clique of $G^c$ corresponds to a maximal independent subset of $G$.

Thus, we run the clique algorithm on $G^c$ to calculate such a set.


\emph{Remark.} Our procedure as implemented in my code chooses for $I_i$ at each stage the first clique found in $G - I_1 - \dots - I_{i-1}$ where the cliques are ordered as the vertices are ordered and lexicographically. For example, if our two cliques are $C_1 = [3,5,7,10]$, $C_2 = [3,5,8,9]$, we say $C_1 < C_2$ as the first index in which $C_1$ and $C_2$ differ, $C_1$ has the lower number. It may be possible that in some graphs, choosing different cliques at each stage may result in a differently sized partition of $G$ into sets. We do not check all possible cliques like this, but if one were interested in an algorithm giving an improved lower bound, albeit with more computations required, implementing this is a possibility.

\subsubsection{A human-verifiable example of the clique algorithm working}

As before, here is a human readable example of the program working, running on $10$ vertices:

\begin{tcolorbox}[size=small]
	\begin{verbatim}
		Question 5
		Finding I1, I2, I3, ... for the following graph:
		VertexOrderedGraph(10, (1, 3) (1, 4) (1, 5) (1, 6) (1, 9) (1, 10) (2, 4) (2, 8)
		(2, 9) (2, 10) (3, 4) (3, 7) (3, 10) (4, 7) (4, 8) (5, 6) (5, 8) (5, 9) (6, 7)
		(6, 9) (7, 8) (8, 10))
		Found:
		I_1 = [1, 2, 7]
		I_2 = [3, 6, 8]
		I_3 = [4, 5, 10]
		I_4 = [9]
	\end{verbatim}
\end{tcolorbox}

It is possible to verify that:

\begin{itemize}
	\item $I_1, I_2, I_3, I_4$ partition the graph into a valid colouring
	\item $I_1$ is a clique in the original graph $G$, indeed it is the first clique in lexicographic order, $I_2$ is the first clique in lexicographic order of $G-I_1$, etc.
 \end{itemize}
 
 i.e. the program is working correctly.
 
 \subsubsection{Comparison of upper bound to results from previous questions}
 
 The following two tables provide ten examples of the algorithm working on $\mathcal{G}(70, 0.5)$ and $\mathcal{G}_7(70, 0.5)$ each. I have had to omit tests on $\mathcal{G}_3(70, 0.75)$ for this algorithm due to the program being unable to complete them in a reasonable time. Recall, however, that the question $1$ methods to calculate an upper bound for such graphs were very effective in that case, so this is not much of a disappointment.
 
 \begin{center}
 	\pgfplotstabletypeset[
 	columns={num, oldUpperBound, newUpperBound, lowerBound},
 	columns/num/.style={column name={Graph tested}, string type},
 	columns/oldUpperBound/.style={column name={Upper bound (Q1 algorithm)}, precision=5},
 	columns/newUpperBound/.style={column name={Upper bound (Q5 algorithm)}, precision=5},
 	columns/lowerBound/.style={column name={Lower bound}, precision=5},
 	every column/.style={column type/.add={|}{}},
 	every last column/.style={column type/.add={}{|}},
 	every head row/.style={before row=\hline, after 	row=\hline},
 	every last row/.style={after row=\hline},
 	]{Program Output/Question5_Table_G.txt}
 	\captionof{table}{Upper and lower bounds on the chromatic number calculated by the question 1, question 5 \& clique algorithms for ten graphs in $\mathcal{G}(70,0.5)$.}
 \end{center}
 
  \begin{center}
 	\pgfplotstabletypeset[
 	columns={num, oldUpperBound, newUpperBound, lowerBound},
 	columns/num/.style={column name={Graph tested}, string type},
 	columns/oldUpperBound/.style={column name={Upper bound (Q1 algorithm)}, precision=5},
 	columns/newUpperBound/.style={column name={Upper bound (Q5 algorithm)}, precision=5},
 	columns/lowerBound/.style={column name={Lower bound}, precision=5},
 	every column/.style={column type/.add={|}{}},
 	every last column/.style={column type/.add={}{|}},
 	every head row/.style={before row=\hline, after 	row=\hline},
 	every last row/.style={after row=\hline},
 	]{Program Output/Question5_Table_G7.txt}
 	\captionof{table}{Upper and lower bounds on the chromatic number calculated by the question 1, question 5 \& clique algorithms for ten graphs in $\mathcal{G}_7(70,0.5)$.}
 \end{center}
 
 As can be seen in both tables, in both cases, question 5's algorithm did better on all samples than the question 1 algorithm.
 
 We may be interested in the value of $\mathbb{E}(\text{number of colours used in Q1}) - \mathbb{E}(\text{number of colours used in Q5})$, i.e. the average number of colours question 5's algorithm saves.
 
The first table gives the estimate that, for $\mathcal{G}(70,0.5)$, this statistic is $1.8$, and the second table gives, for $\mathcal{G}_7(70,0.5)$, the estimate is $4.7$.

Recall from question 2 that it is possible to colour $\mathcal{G}_k(n,p)$ in no more than $k$ colours. The program exploits how we proved this in the $\mathcal{G}_7(70,0.5)$ case, as it is likely that a maximal independent set in the graph corresponds to vertices whose labels are all in the same congruence class $\mod 7$ on most steps of the algorithm. There is room for error here as the algorithm is not guaranteed to do this, hence why we get a bound of $8$ or $9$ in most cases. It is however notable that our results so far were able to classify $\chi(G)$ in $3$ of $10$ cases, and lower the possible range significantly in the others.

\subsubsection{Varying $p$}

For $n=70$ vertices, I ran $5$ tests in each of $p=0.4,0.41,\dots, 0.6$ for both $\mathcal{G} := \mathcal{G}(70,p)$ and $\mathcal{G}_7 := \mathcal{G}_7(70,p)$, finding the average upper bound for both the question 1 and question 5 tests, and the difference between them.

\begin{center}
	\pgfplotstabletypeset[
	columns={p, GoldUpperBound, GnewUpperBound, dGBound, G7oldUpperBound, G7newUpperBound, dG7bound},
	columns/p/.style={column name={$p$}, string type},
	columns/GoldUpperBound/.style={column name={$\mathcal{G}$ avg. (Q1)}, precision=5},
	columns/GnewUpperBound/.style={column name={$\mathcal{G}$ avg. (Q5)}, precision=5},
	columns/dGBound/.style={column name={$\mathcal{G}$ (Q1-Q5)}, precision=5},
	columns/G7oldUpperBound/.style={column name={$\mathcal{G}_7$ avg. (Q1)}, precision=5},
	columns/G7newUpperBound/.style={column name={$\mathcal{G}_7$ avg. (Q5)}, precision=5},
	columns/dG7bound/.style={column name={$\mathcal{G}_7$ (Q1-Q5)}, precision=5},
	every column/.style={column type/.add={|}{}},
	every last column/.style={column type/.add={}{|}},
	every head row/.style={before row=\hline, after 	row=\hline},
	every last row/.style={after row=\hline},
	]{Program Output/Question5_varyingp_70.txt}
	\captionof{table}{Average upper bounds for 5 samples in $\mathcal{G} := \mathcal{G}(70,p)$ and $\mathcal{G}_7 := \mathcal{G}_7(70,p)$ for varying $p$}
\end{center}

In $\mathcal{G}(70,p)$, we can see that the average upper bound on the chromatic number increases for both the question 1 and question 5 algorithm as $p$ increases, which is understandable as the larger number of edges will result in a larger number of colours required. A trend in the difference between the question 1 and question 5 algorithms does not appear to be visible. There is a very slight positive correlation, that is larger $p$ results in a larger difference (Pearson's correlation coefficient is about $0.37$). I have repeated this test for other $n$ as can be seen in the next section, and those results suggest no correlation, so this is likely an outlier. That is, the number of colours that the question 5 algorithm saves over the question 1 algorithm is essentially constant with respect to $p$.

In $\mathcal{G}_7(70,p)$, there is a clear trend: the question 5 algorithm gets much better for larger $p$, resulting in an increasing difference between the algorithms. This makes sense when we consider our reasoning above and in question 2; an optimal colouring of a graph in $\mathcal{G}_7(70,p)$ is where the vertices are coloured with respect to their congruence class $\mod 7$. As remarked in this question, it is likely that the maximal independent sets found correspond to a congruence class, although it is possible a different clique in the graph's compliment is picked. However, as $p$ increases, the chance such a clique forms decreases (as fewer edges exist in the complement), and at $p=1$ the only maximal independent sets that can be picked are the congruence classes. This proves that the expected value of the number of colours used in the question 5 algorithm for a $\mathcal{G}_7(70,p)$ graph tends to $7$ as $p \to 1$, which indeed matches what we find in the table.

\begin{center}
	\begin{tikzpicture}
		\begin{axis}[samples=100,axis lines=middle,
			xlabel=$p$,
			ylabel={Average difference in \#colours used in Q1 vs Q5 algorithms},
			xmin=0.38,
			xmax=0.6,
			xtick={0.4,0.45,...,0.6},
			ymin=0,
			ymax=2.5,
			ytick={0,0.2,...,2.5},
			width=6in,
			height=6cm,
			axis x discontinuity=crunch,
			]
			\addplot[only marks, mark=*, color=blue] table [x=p,y=dGBound] {Program Output/Question5_varyingp_70.txt};
			
		\end{axis}
	\end{tikzpicture}
	\captionof{figure}{The average differences between the number of coloured used in the question 1 algorithm vs the question 5 algorithm for graphs in $\mathcal{G}(70,p)$.}
\end{center}

\begin{center}
	\begin{tikzpicture}
		\begin{axis}[samples=100,axis lines=middle,
			xlabel=$p$,
			ylabel={Average difference in \#colours used in Q1 vs Q5 algorithms},
			xmin=0.38,
			xmax=0.6,
			xtick={0.4,0.45,...,0.6},
			ymin=0,
			ymax=7.5,
			ytick={0,0.5,...,7.5},
			width=6in,
			height=6cm,
			axis x discontinuity=crunch,
			]
			\addplot[only marks, mark=*, color=blue] table [x=p,y=dG7bound] {Program Output/Question5_varyingp_70.txt};
			
		\end{axis}
	\end{tikzpicture}
	\captionof{figure}{The average differences between the number of coloured used in the question 1 algorithm vs the question 5 algorithm for graphs in $\mathcal{G}_7(70,p)$.}
\end{center}

\newpage

\subsubsection{Varying $n$ and $p$}

The amount of time my program takes to compute these results increases rapidly as $n$ increases. So, I have investigated for $30 \leq n \leq 70$.

I have constructed tables such as the one above for $n=30,35,40,\dots,70$ which can be found in the output in the appendix. We investigate the trends here for the question 5 algorithm.



%\pgfplotstabletypeset[color cells]
%\\
\begin{center}
	\pgfplotstabletypeset[
	columns/p/.style={column name={$\bsfrac{n}{p}$}, precision=5},
	every first column/.style={column type/.add={}{|}},
	every head row/.style={after 	row=\hline}]{Program Output/Question5_G_heatmap.txt}
	
	\captionof{table}{Average number of colours used for the question $5$ algorithm on $\mathcal{G}(n,p)$, 5 samples used for each cell.}
\end{center}

As can be seen, the average number of colours used increases as either $n$ or $p$ increases, which is to be expected as per our previous explanations.

It appears, that for each fixed $n$, the expected number of colours increases linearly with $p$ (it can be checked that for each fixed $n$, the correlation coefficient is $>0.85$, with most very close to $1$). Also, for fixed $p$, the expected number of colours increases linearly with $n$ (it can be checked that for each fixed $p$, the correlation coefficient is $>0.9$, with most very close to $1$).

Thus, it seems reasonable to conjecture that $$\mathbb{E}(\text{\#colours the question 5 algorithm takes to colour a graph in $\mathcal{G}(n,p)$}) \propto np \quad \forall n \forall p.$$

\begin{center}
	\pgfplotstabletypeset[
	columns/p/.style={column name={$\bsfrac{n}{p}$}, precision=5},
	every first column/.style={column type/.add={}{|}},
	every head row/.style={after 	row=\hline}]{Program Output/Question5_G7_heatmap.txt}
	
	\captionof{table}{Average number of colours used for the question $5$ algorithm on $\mathcal{G}_7(n,p)$, 5 samples used for each cell.}
\end{center}

The trends for $\mathcal{G}_7(n,p)$ appear much less clear, with the expected value appearing to increase as $p$ increases for small $n$, and decrease as $p$ increases for large $n$, within the table. As per our reasoning earlier, we know that as $p \to 1$, the expected value tends to $7$, and this is independent of $n$, so the increasing trends cannot continue.

What appears to be the case is that for a given $n$ there is some $p_n$ that is least optimal, that is, the question 5 algorithm takes the most colours on average for $p \approx p_n$, being more efficient for $p < p_n$, and $p > p_n$.

For example, with $n=45$, we appear to see the maximum within the table, it seems the number of colours used peaks somewhere around $0.55$, that is, $p_{45} \approx 0.55$.

This trend can be explained by the fact that there is a trade-off:

\begin{itemize}
	\item For $p$ small, few edges exist in the average graph picked from $\mathbb{G}(n,p)$, thus any maximal independent sets picked for $I_1, I_2, \dots$ are likely to be efficient.
	\item For $p$ large, as discussed, we are more likely to be forced into picking $I_1, I_2, \dots$ corresponding to the congruence classes $\mod 7$ of the vertices' labels, thus forcing us into what is almost surely a minimal colouring.
	\item For $n$ small, we are less likely to exhaust as many colours due to their being fewer vertices to colour, so the positives of $p$ being small as described here is more likely to be pronounced. This explains why these values peak for larger $p$ in the table.
	\item For $n$ large, we are likely to exhaust more colours, so the positives of large $p$ locking us into the specific colouring corresponding to congruence classes is more likely to be effective. This explains why these values peak for earlier $p$ in the table.
\end{itemize}

For fixed $p$, the expected value seems to vary linearly with $n$, in particular for $p$ not close to $0.5$. The correlation coefficient for such $p$ is high.



\newpage

\newpage
				
\section{Appendix A: Code}
\label{appendix_a_code}

\lstset{frame=tb,
	language=[Sharp]C,
	aboveskip=3mm,
	belowskip=3mm,
	showstringspaces=false,
	columns=flexible,
	basicstyle={\small\ttfamily},
	numbers=none,
	numberstyle=\tiny\color{gray},
	keywordstyle=\color{blue},
	commentstyle=\color{dkgreen},
	stringstyle=\color{mauve},
	breaklines=true,
	breakatwhitespace=true,
	tabsize=3
}
				
\subsection{Matrices.cs}

\begin{lstlisting}
	using System;
	using System.Collections.Generic;
	
	namespace Matrices
	{
		public class Matrix
		{
			
			public static Random random = new Random();
			private int[] underlyingArray;
			
			private int rows;
			private int columns;
			
			public Matrix(int height, int width)
			{
				underlyingArray = new int[width * height];
				
				rows = height;
				columns = width;
			}
			
			public int this[int row, int column]
			{
				get
				{
					return underlyingArray[(column - 1) * rows + row - 1];
				}
				set
				{
					underlyingArray[(column - 1) * rows + row - 1] = value;
				}
			}
			
			public override string ToString()
			{
				string lineBreak = "|";
				string str = "[ ";
				for (int i = 1; i <= Rows; i++)
				{
					for (int j = 1; j <= Columns; j++)
					{
						str += this[i, j] + " ";
						
					}
					
					if (i != Rows)
					{
						str += lineBreak + " ";
					}
					
				}
				str += "]";
				return str;
			}
			
			public (int, int) size
			{
				get
				{
					return (rows, columns);
				}
			}
			
			public int Rows
			{
				get
				{
					return rows;
				}
			}
			
			public int Columns
			{
				get
				{
					return columns;
				}
			}
			
			public static Matrix operator *(Matrix A, Matrix B) //multiply 2 matrices together
			{
				
				if (A.columns != B.rows)
				{
					throw new Exception("Cannot multiply a " + A.size + " matrix with a " + B.size + " matrix.");
				}
				
				Matrix AB = new Matrix(A.rows, B.columns);
				
				for (int i = 1; i <= AB.Rows; i++)
				{
					for (int j = 1; j <= AB.Columns; j++)
					{
						AB[i, j] = 0;
						
						for (int k = 1; k <= A.columns; k++)
						{
							AB[i, j] += A[i, k] * B[k, j];
						}
						
					}
				}
				
				return AB;
			}
			
			public static Matrix operator *(Matrix A, Vector B) //multiply a matrix with a vector
			{
				return A * B.AsMatrix();
			}
			
			public Matrix Mod(int number) //returns this matrix mod some number
			{
				Matrix newMatrix = new Matrix(this.Rows, this.Columns);
				
				for (int i = 1; i <= this.Rows; i++)
				{
					for (int j = 1; j <= this.Columns; j++)
					{
						newMatrix[i, j] = this[i, j] % number;
						
						while (newMatrix[i, j] < 0) //the % in C# may not work properly if negative, so we need to increase until positive.
						{
							newMatrix[i, j] += number;
						}
					}
				}
				
				return newMatrix;
			}
			
			public Vector[] RowsAsVectors()
			{
				List<Vector> rowVectors = new List<Vector>();
				for (int i = 1; i <= this.Rows; i++)
				{
					Vector vector = new Vector(this.Columns);
					for (int j = 1; j <= this.Columns; j++)
					{
						vector[j] = this[i, j];
					}
					rowVectors.Add(vector);
				}
				return rowVectors.ToArray();
			}
			
			public Matrix Transpose()
			{
				Matrix transpose = new Matrix(this.Columns, this.Rows);
				
				for (int i = 1; i <= this.Rows; i++)
				{
					for (int j = 1; j <= this.Columns; j++)
					{
						transpose[j, i] = this[i, j];
					}
				}
				return transpose;
			}
			
			public static Matrix RandomMatrix(int inPrimeField, int width, int height)
			{
				Matrix matrix = new Matrix(width, height);
				for (int i = 1; i <= width; i++)
				{
					for (int j = 1; j <= height; j++)
					{
						matrix[i, j] = random.Next(0, inPrimeField);
					}
				}
				return matrix;
			}
			
			/// <summary>
			/// Makes a nice LaTeX string for this matrix
			/// </summary>
			/// <returns></returns>
			public string ToLaTeX()
			{
				
				string result = "";
				result += "\\begin{pmatrix}\n";
				for (int i = 1; i <= this.Rows; i++)
				{
					string line = "";
					for (int j = 1; j <= this.Columns; j++)
					{
						int value = this[i, j];
						
						line += value;
						if (j == this.Columns)
						{
							if (i != this.Rows)
							{
								line += " \\\\";
							}
						}
						else
						{
							line += " & ";
						}
						
					}
					result += line;
					result += "\n";
				}
				result += "\\end{pmatrix}";
				return result;
			}
			
			public Matrix transpose
			{
				get
				{
					Matrix Transpose = new Matrix(Columns, Rows);
					
					for (int i = 1; i <= Transpose.Rows; i++)
					{
						for (int j = 1; j <= Transpose.Columns; j++)
						{
							Transpose[i, j] = this[j, i];
						}
					}
					
					return Transpose;
				}
			}
			
			public bool AllEntriesZero()
			{
				foreach (int entry in underlyingArray)
				{
					if (entry != 0)
					{
						return false;
					}
				}
				return true;
			}
		}
		
		public class Vector
		{
			Matrix underlyingMatrix;
			
			public Vector(int size)
			{
				underlyingMatrix = new Matrix(size, 1);
			}
			
			public Vector(Matrix matrix) //turn some matrix into a vector, if possible.
			{
				if (matrix.Rows == 1)
				{
					underlyingMatrix = matrix.Transpose();
				}
				else if (matrix.Columns == 1)
				{
					underlyingMatrix = matrix;
				}
				else
				{
					throw new Exception("Given matrix is not either a row or column vector so cannot convert into vector object.");
				}
				
			}
			
			public int this[int entry]
			{
				get
				{
					return underlyingMatrix[entry, 1];
				}
				set
				{
					underlyingMatrix[entry, 1] = value;
				}
			}
			
			public int Size
			{
				get
				{
					return underlyingMatrix.Rows;
				}
			}
			
			public Matrix AsMatrix()
			{
				return underlyingMatrix;
			}
			
			public Vector Mod(int number) //returns this vector mod some number
			{
				return new Vector(AsMatrix().Mod(number));
			}
			
			public override string ToString()
			{
				return underlyingMatrix.ToString();
			}
			
			
			public string ToLaTeX()
			{
				return underlyingMatrix.ToLaTeX();
			}
			
			/// <summary>
			/// Returns the last non-zero index of the vector. E.g. (1,1,0,1,0,0) returns 4 as the 4th index is non-zero, all others are zero.
			/// If it is the zero vector, returns 0
			/// </summary>
			/// <returns></returns>
			public int LastNonZeroIndex()
			{
				int lastIndex = 0;
				for (int i = 1; i <= this.Size; i++) { 
					
					if (this[i] != 0)
					{
						lastIndex = i;
					}
					
				}
				return lastIndex;
			}
			
		}
	}
	
	
\end{lstlisting}

\subsection{MatricesKernel.cs}

\begin{lstlisting}
	using System;
	using System.Collections.Generic;
	using System.Linq;
	using System.Text;
	using ContinuedFractionProject;
	using Matrices;
	using Matrices.RREF;
	
	namespace Matrices.Kernel
	{
		public static class MatricesKernel
		{
			/*
			* This class provides extensions to the matrices that allow the calculation of the kernel, similarly to what was done in Q3 with the RREF.
			*/
			
			public static Vector[] GetBasisOfKernel(this Matrix matrix, int prime)
			{
				
				if (matrix.AllEntriesZero()) //special case, this algorithm doesnt work.
				{
					
					List<Vector> vectors = new List<Vector>();
					
					for (int i = 1; i <= matrix.Columns; i++) {
						Vector v = new Vector(matrix.Columns);
						
						v[i] = 1;
						vectors.Add(v);
					}
					return vectors.ToArray();
				}
				//Console.WriteLine("Going to find the RREF");
				//Get the RREF and its ls
				Matrix RREF = matrix.GetReducedRowEchelonForm(prime);
				
				//Console.WriteLine("Found RREF");
				
				int m = RREF.Rows;
				int n = RREF.Columns;
				int r = RREF.LastNonZeroRow();
				//Console.WriteLine("n = " + n);
				//Console.WriteLine("m = " + m);
				//Console.WriteLine("r = " + r);
				
				int[] ls = RREF.GetRREFLValues(); //the values of l(1), ..., l(r). due to array indexing we store l(1) at 0, etc.
				
				//we need to construct a matrix B such that the x_{l(i)} are expressed in terms of the other x_j.
				//see write up in document - this is how we do it:
				
				Matrix XRelationMatrix = new Matrix(r, n - r); //relates the xs to each other.
				//Console.WriteLine(XRelationMatrix.size);
				
				for (int i = 1; i <= r; i++)
				{
					//Console.WriteLine("l(" + i + ") = " + ls[i - 1]);
					int newIndex = 1;
					
					for (int j = 1; j <= n; j++)
					{
						if (ls.Contains(j) == false) //this is not one of the x_{l(i)}
						{
							//Console.WriteLine(j);
							XRelationMatrix[i, newIndex] = -RREF[i, j]; //negative due to rearrangement to other side of the equation: vector of x_l(i)s = B x_others
							newIndex++;
						}
					}
				}
				
				//Console.WriteLine(XRelationMatrix.ToString());
				
				//we have the required matrix.
				
				//now we can work out the basis.
				
				Vector[] basis = new Vector[n - r];
				
				for (int vectorComponentToSetToOne = 1; vectorComponentToSetToOne <= n - r; vectorComponentToSetToOne++)
				{
					int xToSetToOne = 1;
					int lsFound = 0;
					if (ls.Contains(xToSetToOne)) //this is not one of the x_{l(i)}
					{
						lsFound++;
					}
					
					while (xToSetToOne - lsFound != vectorComponentToSetToOne)
					{
						xToSetToOne++;
						if (ls.Contains(xToSetToOne)) //this is not one of the x_{l(i)}
						{
							lsFound++;
						}
					}
					
					//construct the vector of the xjs
					
					Vector xjsVector = new Vector(n - r); //defaults to all 0 so set required component to 1
					xjsVector[vectorComponentToSetToOne] = 1;
					
					Vector valuesOfLs = new Vector(XRelationMatrix * xjsVector);
					
					valuesOfLs = valuesOfLs.Mod(prime); //mod it in case of any overflow out of the range 0, ... p-1
					
					//now construct the basis vector
					
					//set the non l x to 1
					Vector basisVector = new Vector(n);
					basisVector[xToSetToOne] = 1;
					
					//set the l xs to whatever value calculated.
					for (int i = 1; i <= ls.Length; i++)
					{
						int l = ls[i - 1]; //off by 1 due to array indexing
						basisVector[l] = valuesOfLs[i];
					}
					basisVector = basisVector.Mod(prime); //mod it in case of any overflow out of the range 0, ... p-1 (may happen due to negatives)
					
					basis[vectorComponentToSetToOne - 1] = basisVector; //done!
				}
				
				//just as a check, make sure to remove all zero vectors. shouldnt have any any way.
				
				List<Vector> basisFinal = new List<Vector>();
				
				foreach (Vector vector in basis)
				{
					if (vector.LastNonZeroIndex() != 0)
					{
						basisFinal.Add(vector);
					}
				}
				
				return basisFinal.ToArray();
			}
		}
	}
	
\end{lstlisting}

\subsection{MatricesRREF.cs}

\begin{lstlisting}
	using System;
	using ContinuedFractionProject;
	using Matrices;
	using PrimeFieldInverses;
	
	namespace Matrices.RREF
	{
		public static class MatricesRREF
		/*
		* This class provides extensions to the matrices that allow the calculation of the Reduced Row Echelon Form.
		*/
		{
			
			public static int LastNonZeroRow(this Matrix matrix)
			{
				//returns the number of the last row which is non zero.
				
				int currentRowTesting = matrix.Rows; //start at the bottom
				
				bool rowIsAllZeroes = true;
				
				do
				{
					for (int j = 1; j <= matrix.Columns; j++)
					{
						if (matrix[currentRowTesting, j] != 0) //this row is not completely made of 0s
						{
							rowIsAllZeroes = false;
						}
					}
					if (rowIsAllZeroes) //this row we have just tested is all 0s. check the next highest one.
					{
						currentRowTesting--;
						
						if (currentRowTesting == 0) //we have gone before the first row - meaning this matrix is the zero matrix, a special case. say it is in RREF.
						{
							return matrix.Rows;
						}
					}
				} while (rowIsAllZeroes);
				
				//the row we have just tested has some non zero element so we can work out what r is now
				
				return currentRowTesting;
			}
			public static bool IsInReducedRowEchelonForm(this Matrix matrixToTest, int prime)
			{
				
				/*
				* CONDITION 1: "For some r..."
				*/
				
				int r = matrixToTest.LastNonZeroRow();
				/*
				* CONDITION 2: "For each i..."
				*/
				
				int[] ls = new int[r]; //the values of l(1), ..., l(r). due to array indexing we store l(1) at 0, etc.
				
				for (int i = 1; i <= r; i++)
				{
					//calculate l(i)
					
					int j = 1;
					
					while (matrixToTest[i, j] == 0)
					{
						j++;
						if (j > matrixToTest.Columns) //the whole row was zeroes (not at the bottom but somewhere in the middle)
						{
							return false;
						}
					}
					
					if (matrixToTest[i, j] != 1) //the row does not start with a 1 after the 0s so this is not in the form.
					{
						return false;
					}
					else //weve found l(i)
					{
						ls[i - 1] = j;
					}
				}
				
				//we have all the values of l.
				
				/*
				* CONDITION 3: "l(1) < l(2) < ..."
				*/
				
				//check to make sure the ls obey this.
				
				int last = ls[0];
				
				for (int i = 1; i < r; i++)
				{
					if (ls[i] <= last) //fails this condition
					{
						return false;
					}
					last = ls[i];
				}
				
				/*
				* CONDITION 4: "For each k..."
				*/
				for (int k = 2; k <= r; k++)
				{
					int j = ls[k - 1];
					
					for (int i = 1; i < k; i++)
					{
						if (matrixToTest[i, j] != 0) //fails the test.
						{
							return false;
						}
					}
				}
				
				//meets all 4 conditions so this is in RREF.
				
				return true;
			}
			
			public static int[] GetRREFLValues(this Matrix matrixInRREF)
			{
				//returns the values of l(1), ... for some matrix in RREf.
				int[] ls = new int[matrixInRREF.LastNonZeroRow()]; //the values of l(1), ..., l(r). due to array indexing we store l(1) at 0, etc.
				
				for (int i = 1; i <= matrixInRREF.LastNonZeroRow(); i++)
				{
					//calculate l(i)
					
					int j = 1;
					
					while (matrixInRREF[i, j] == 0)
					{
						j++;
					}
					
					if (matrixInRREF[i, j] != 1) //the row does not start with a 1 after the 0s so this is not in the form.
					{
						throw new Exception("This matrix is not in RREF.");
					}
					else //weve found l(i)
					{
						ls[i - 1] = j;
					}
				}
				return ls;
			}
			
			//Below we code the operations T, D, S that we will need for GE.
			public static Matrix TransposeRows(this Matrix matrix, int i, int j, int prime)
			{
				Matrix newMatrix = matrix.Mod(prime);
				
				//new matrix is currently a copy of the original matrix, modded. perform the transposition
				
				for (int column = 1; column <= matrix.Columns; column++)
				{
					newMatrix[i, column] = matrix[j, column];
					newMatrix[j, column] = matrix[i, column];
				}
				
				return newMatrix;
			}
			
			public static Matrix Divide(this Matrix matrix, int i, int a, int prime)
			{
				//to divide, we need to get the inverses of this prime
				
				
				if (a == 0)
				{
					throw new Exception("Cannot divide by zero.");
				}
				Matrix newMatrix = matrix.Mod(prime);
				
				//new matrix is currently a copy of the original matrix, modded. perform the division
				
				for (int column = 1; column <= matrix.Columns; column++)
				{
					newMatrix[i, column] *= Inverses.InverseOf(a, prime); //multiplying by the inverse same as dividing.
				}
				
				newMatrix = newMatrix.Mod(prime); //mod the matrix after applying the operation in case anything has gone out of the range.
				return newMatrix;
			}
			
			public static Matrix Subtract(this Matrix matrix, int i, int a, int j, int prime)
			{
				
				if (i == j) //we require i =/= j.
				{
					throw new Exception("Cannot allow the subtraction of a row from itself.");
				}
				Matrix newMatrix = matrix.Mod(prime);
				
				//new matrix is currently a copy of the original matrix, modded. perform the transposition
				
				for (int column = 1; column <= matrix.Columns; column++)
				{
					newMatrix[i, column] -= a * newMatrix[j, column];
				}
				
				newMatrix = newMatrix.Mod(prime); //mod the matrix after applying the operation in case anything has gone out of the range.
				return newMatrix;
			}
			
			
			public static Matrix GetReducedRowEchelonForm(this Matrix matrix, int prime) //get RREF for this matrix.
			{
				
				if (matrix.AllEntriesZero()) //technically not in RREF, but we cannot put it into such. just return it, if we need to deal with zero matrices we do it elsewhere
				{
					return matrix;
				}
				if (IsInReducedRowEchelonForm(matrix, prime)) //we are done
				{
					return matrix;
				}
				
				Matrix newMatrix = matrix.Mod(prime);
				
				int currentColumn = 1; //the current column that we shall intend to modify into the form it should be in for RREF.
				int currentRowToLead = 1;
				
				
				while ((currentColumn > newMatrix.Columns || currentRowToLead > newMatrix.Rows) == false && IsInReducedRowEchelonForm(newMatrix, prime) == false) //when this condition is true, we are at the end of the process since we have gone beyond the edge of the matrix. or if in RREF before, just quit since solved.
				{
					//Find some row in this column that we are sorting out that has a non-zero number, if any exists.
					
					int rowWithNonZeroInCurrentColumn = 0;
					
					for (int i = currentRowToLead; i <= newMatrix.Rows; i++)
					{
						if (newMatrix[i, currentColumn] != 0)
						{
							rowWithNonZeroInCurrentColumn = i;
							break;
						}
					}
					
					if (rowWithNonZeroInCurrentColumn == 0) //this column is full of nothing but zeroes, so it is fine to skip over trying to fix it. there is no new top row though
					{
						currentColumn++;
					}
					else
					{
						
						//move this row that we have just found to the top of the matrix (excluding previously sorted out rows)
						newMatrix = newMatrix.TransposeRows(rowWithNonZeroInCurrentColumn, currentRowToLead, prime);
						//we want the leading coefficient to be 1. we can do that by dividing by whatever the current leading coeff is.
						newMatrix = newMatrix.Divide(currentRowToLead, newMatrix[currentRowToLead, currentColumn], prime);
						//we want every other row to have a 0 in this column, so we can do this by applying subtraction.
						//we do not need to worry about messing up any of the previous columns above since by previous steps in this algorithm we have everything before the leading 1 in this column a 0.
						
						for (int i = 1; i <= matrix.Rows; i++)
						{
							if (i != currentRowToLead) //cant substract from this same row.
							{
								newMatrix = newMatrix.Subtract(i, newMatrix[i, currentColumn], currentRowToLead, prime); //subtracting this amount from the row will create a 0 since there is a leading 1
							}
						}
						
						
						//we have sorted out this row and column. move onto the next ones
						currentColumn++;
						currentRowToLead++;
					}
					
					
				}
				
				
				
				
				
				if (IsInReducedRowEchelonForm(newMatrix, prime) == false) //something has gone wrong, this should not occur.
				{
					throw new Exception("Error in algorithm - does not satisfy RREF conditions");
				}
				
				return newMatrix;
			}
		}
	}
	
\end{lstlisting}

\subsection{ArrayExtensions.cs}

\begin{lstlisting}
	using System;
	using System.Collections.Generic;
	using System.Linq;
	using System.Text;
	using System.Threading.Tasks;
	
	namespace ContinuedFractionProject
	{
		public static class ArrayExtensions
		{
			public static string ToFormattedString<T>(this T[] thisArray)
			{
				if (thisArray.Length == 0)
				{
					return "[]";
				}
				string result = "[";
				
				foreach (T item in thisArray)
				{
					if (item is null)
					{
						result += "null";
					}
					else
					{
						result += item.ToString();
					}
					
					result += ", ";
				}
				
				result = result.Substring(0, result.Length - 2);
				result += "]";
				return result;
			}
			
			/// <summary>
			/// Remove all copies of a given item.
			/// </summary>
			/// <typeparam name="T"></typeparam>
			/// <param name="array"></param>
			/// <param name="itemToRemove"></param>
			/// <returns></returns>
			public static T[] RemoveAll<T>(this T[] array, T itemToRemove)
			{
				List<T> newArr = new List<T>();
				
				foreach (T item in array)
				{
					if (item.Equals(itemToRemove) == false)
					{
						newArr.Add(item);
					}
				}
				return newArr.ToArray();
			}
			
			/// <summary>
			/// Gets all elements that appear in the array more than once. Each more-than-once-appearing element will appear in the new array one time.
			/// </summary>
			/// <param name="array"></param>
			/// <example>{1, 2, 2, 3} returns {2}</example>
			/// <returns></returns>
			public static T[] AllEntriesThatAppearMoreThanOnce<T>(this T[] array)
			{
				List<T> duplicated = new List<T>();
				
				
				for (int i = 0; i < array.Length; i++)
				{
					T item = array[i];
					if (duplicated.Contains(item) == false) //we dont already know this one is duplicated, i.e. its the first one of its type
					{
						for (int j = i+1; j < array.Length; j++)
						{
							T item2 = array[j];
							if (item.Equals(item2))
							{
								duplicated.Add(item);
								break;
								
							}
						}
					}
				}
				return duplicated.ToArray();
			}
			
			/// <summary>
			/// Returns the indices of the array matching a given item.
			/// </summary>
			/// <typeparam name="T"></typeparam>
			/// <param name="arr"></param>
			/// <param name="item"></param>
			/// <returns></returns>
			public static int[] IndicesOf<T>(this T[] arr, T item)
			{
				List<int> indices = new List<int>();
				for (int i = 0; i < arr.Length;i++)
				{
					if (arr[i].Equals(item))
					{
						indices.Add(i);
					}
				}
				return indices.ToArray();
			}
		}
	}
	
\end{lstlisting}

\subsection{ContinuedFractions.cs}

\begin{lstlisting}
	using System;
	using System.Collections.Generic;
	using System.Linq;
	using System.Runtime.CompilerServices;
	using System.Text;
	using System.Threading.Tasks;
	
	namespace ContinuedFractionProject
	{
		public static class ContinuedFractions
		{
			
			public static int PeriodOfContinuedFractionExpansionOfSquareRoot(this ulong integerUnderRoot)
			{
				List<(ulong, ulong)> foundRSValues = new List<(ulong, ulong)>(); 
				
				List<ulong> AList = new List<ulong>();
				
				bool found = false;
				
				//start with (0 + sqrt(N))/1
				ulong rCurrent = 0;
				ulong sCurrent = 1;
				
				int i = 0;
				while (true) //this will eventually terminate by itself
				{
					
					
					//get the new values
					
					double convergentCurrent = ((double)rCurrent + Math.Sqrt(integerUnderRoot)) / (double)sCurrent;
					ulong floorCurrent = (ulong)Math.Floor(convergentCurrent); //a_n
					
					
					if (i > 0)
					{
						//add current r and s
						foundRSValues.Add((rCurrent, sCurrent));
						
						//add a
						AList.Add(floorCurrent);
					}
					
					
					if (convergentCurrent == floorCurrent) //xn = an so terminate, its a square number
					{
						return 0;
					}
					
					ulong rNext = floorCurrent * sCurrent - rCurrent; //as-r
					ulong sNext = (integerUnderRoot - (rNext) * (rNext)) / sCurrent; //(N-(r-as)^2) / s
					
					if (foundRSValues.Contains((rNext, sNext))) //were looping
					{
						return i;
					}
					
					rCurrent = rNext;
					sCurrent = sNext;
					i++;
					
				}
			}
			
			/// <summary>
			/// Returns the CFE of a given square root of an integer.
			/// </summary>
			/// <param name="integerUnderRoot">N in the program</param>
			/// <param name="RSValues">The values of r and s as defined in question 2 for each x_n.</param>
			/// <returns></returns>
			public static ulong[] ContinuedFractionExpansionOfSquareRoot(this ulong integerUnderRoot, out (ulong, ulong)[] RSValues, int upToConvergent)
			{
				List<(ulong, ulong)> RSValuesList = new List<(ulong, ulong)>();
				
				List<ulong> AList = new List<ulong>();
				
				//start with (0 + sqrt(N))/1
				ulong rCurrent = 0;
				ulong sCurrent = 1;
				for (int i = 0; i <= upToConvergent; i++)
				{
					
					
					//get the new values
					
					double convergentCurrent = ((double)rCurrent + Math.Sqrt(integerUnderRoot)) / (double)sCurrent;
					ulong floorCurrent = (ulong)Math.Floor(convergentCurrent); //a_n
					
					
					//add current r and s
					RSValuesList.Add((rCurrent, sCurrent));
					
					//add a
					AList.Add(floorCurrent);
					
					
					if (convergentCurrent == floorCurrent) //xn = an so terminate
					{
						break;
					}
					
					ulong rNext = floorCurrent * sCurrent - rCurrent; //as-r
					ulong sNext = (integerUnderRoot - (rNext) * (rNext)) / sCurrent; //(N-(r-as)^2) / s
					
					rCurrent = rNext;
					sCurrent = sNext;
					
					
				}
				
				RSValues = RSValuesList.ToArray();
				
				return AList.ToArray();
				
			}
			
			/// <summary>
			/// Returns a list of partial quotients as tuples, with the first entry corresponding to p, the second corresponding to q. Likely will cause overflow if represents the partial quotients for sqrt(N) with large N, and given array is long.
			/// </summary>
			/// <param name="arr"></param>
			/// <returns></returns>
			public static (ulong, ulong)[] PartialQuotients(this ulong[] arr)
			{
				return PartialQuotientsModulo(arr, 0);
			}
			
			/// <summary>
			/// Returns a list of partial quotients as tuples, with the first entry corresponding to p, the second corresponding to q, modulo a given number.
			/// </summary>
			/// <param name="arr"></param>
			/// <param name="mod">What number to work out the partial quotients mod. put mod = 0 for no modding. Likely to cause overflows in this case</param>
			/// <returns></returns>
			public static (ulong, ulong)[] PartialQuotientsModulo(this ulong[] arr, ulong mod)
			{
				List<(ulong, ulong)> partialQuotientsList = new List<(ulong, ulong)>();
				
				//p, q at index i-2, start at i=0
				ulong P_iMinusTwo = 0;
				ulong Q_iMinusTwo = 1;
				
				//p, q at index i-1, start at i=0
				ulong P_iMinusOne = 1;
				ulong Q_iMinusOne = 0;
				
				int i = 0;
				
				while (i < arr.Length)
				{
					//recursion relations
					ulong P_i, Q_i;
					
					if (mod == 0)
					{
						P_i = (ulong)arr[i] * P_iMinusOne + P_iMinusTwo;
						Q_i = (ulong)arr[i] * Q_iMinusOne + Q_iMinusTwo;
					}
					else
					{
						P_i = ModularArithmetic.Mod(ModularArithmetic.ModMult(arr[i], P_iMinusOne, mod) + P_iMinusTwo, mod);
						Q_i = ModularArithmetic.Mod(ModularArithmetic.ModMult(arr[i], Q_iMinusOne, mod) + Q_iMinusTwo, mod);
					}
					
					
					partialQuotientsList.Add((P_i, Q_i));
					
					//go on to next i
					P_iMinusTwo = P_iMinusOne;
					Q_iMinusTwo = Q_iMinusOne;
					
					P_iMinusOne = P_i;
					Q_iMinusOne = Q_i;
					
					i++;
				}
				
				return partialQuotientsList.ToArray();
			}
			
			/// <summary>
			/// Returns P_n mod N, P_n^2 mod N for a given N and n in a given range
			/// </summary>
			/// <returns>An array of tuples, with the nth entry in the array giving (P_n mod N, P_n^2 mod N)</returns>
			public static (ulong, ulong)[] ConvergentNumeratorsModulo(ulong N, int upToConvergent)
			{
				(ulong, ulong)[] unused = new (ulong, ulong)[0];
				ulong[] aValues = ContinuedFractionExpansionOfSquareRoot(N, out unused, upToConvergent);
				
				ulong[] PValues = aValues.PartialQuotientsModulo(N).Select(x => x.Item1).ToArray();
				
				List<(ulong, ulong)> convergentModValues = new List<(ulong, ulong)>();
				
				foreach (ulong PValue in PValues)
				{
					ulong mod = ModularArithmetic.Mod(PValue, N);
					ulong modSquared = ModularArithmetic.ModMult(mod, mod, N);
					
					convergentModValues.Add((mod, modSquared));
				}
				
				return convergentModValues.ToArray();
			}
			
		}
	}
	
\end{lstlisting}

\subsection{GenericExtensions.cs}

\begin{lstlisting}
	using System;
	using System.Collections.Generic;
	using System.Linq;
	using System.Text;
	using System.Threading.Tasks;
	
	namespace ContinuedFractionProject
	{
		public static class GenericExtensions
		{
			private static Random random = new Random();
			
			/// <summary>
			/// Returns a random integer with a given number of digits
			/// </summary>
			/// <param name="numberOfDigits"></param>
			/// <returns></returns>
			public static long RandomIntegerWithGivenDigits(int numberOfDigits)
			{
				if (numberOfDigits < 1)
				{
					throw new Exception("Need to be given a positive integer for given digits");
				}
				
				
				string result = "";
				
				for (int i = 1; i <= numberOfDigits; i++)
				{
					if (i == 1)
					{
						result += random.Next(1, 10).ToString(); //10 since exclusive upper bound
					}
					else
					{
						result += random.Next(0, 10).ToString(); //10 since exclusive upper bound
					}
					
				}
				
				return long.Parse(result);
			}
			
			public static long[] DistinctRandomIntegersWithGivenDigits(int numberOfDigits, int sampleSizeMax)
			{
				
				long min = (int)Math.Pow(10, numberOfDigits - 1);
				long max = (int)Math.Pow(10, numberOfDigits) - 1;
				
				long totalNumbersInRange = max - min + 1;
				
				long actualSampleSizeMax = sampleSizeMax;
				if (sampleSizeMax > totalNumbersInRange)
				{
					actualSampleSizeMax = totalNumbersInRange;
				}
				
				List<long> result = new List<long>();
				
				for (int i = 0; i < actualSampleSizeMax; i++)
				{
					long numGenerated = 0;
					
					do
					{
						numGenerated = RandomIntegerWithGivenDigits(numberOfDigits);
					} while (result.Contains(numGenerated));
					
					result.Add(numGenerated);
				}
				
				return result.ToArray();
			}
			
			public static long GCD(long a, long b)
			{
				//Console.WriteLine("GCD(" + a + ", " + b + ")");
				//Console.ReadLine();
				if (b > a)
				{
					return GCD(a, b);
				}
				else
				{
					if (b == 0)
					{
						return a;
					}
					else
					{
						return GCD(b, a % b);
					}
				}
			}
		}
	}
	
\end{lstlisting}

\subsection{ModularArithmetic.cs}

\begin{lstlisting}
	using System;
	using System.Collections.Generic;
	using System.Linq;
	using System.Text;
	using System.Threading.Tasks;
	
	namespace ContinuedFractionProject
	{
		/// <summary>
		/// Performs modular arithmetic.
		/// </summary>
		public static class ModularArithmetic //Reused from my 15.1 Primality tests project with some other additions. May contain methods not relevant to this project.
		{
			
			#region Modulus methods
			/// <summary>
			/// Calculates a^b mod N
			/// </summary>
			public static int ModExp(int a, int b, int N)
			{
				if (b < 0)
				{
					throw new Exception("Exponent must be positive.");
				}
				if (a == 0 && b == 0)
				{
					throw new Exception("0^0 not defined.");
				}
				
				
				/*
				* In order to make sure we dont get overflows, we multiply by a and do the mod at each step.
				* This ensures that we never have result be larger than a*(N-1) (result is never more than
				* N-1 after each for loop, so it could be on a*(N-1). In this project we dont work with N > 10^10
				* so we can work with a up to about 10^5 which is more than enough for our purpouses, we wont really
				* go above bases in double digits).
				*/
				
				/*
				* We dont just do a*a* ... * a, b times, since for b large (which we will need since we exponent to the prime)
				* this takes ages. Instead heres a better way:
				* Write b in binary, and calculate a, then a^2, then a^4 by a^2 * a^2, up to the largest binary digit in b.
				* Multiply all together where we have it. E.g. if we want to find a^42, 42 is 101010 in binary so we
				* do a^32 * a^8 * a^2.
				*/
				
				int result = 1;
				
				int newA = a;
				int newB = b;
				newA = Mod(newA, N);
				
				while (newB > 0)
				{
					if (newB % 2 == 1)
					result = ModMult(result, newA, N);
					
					newB /= 2;
					newA = ModMult(newA, newA, N);
					
					
				}
				//Console.WriteLine(a + "^" + b + " = " + result + " mod " + N);
				return result;
			}
			
			/// <summary>
			/// Calculates a^b mod N
			/// </summary>
			public static ulong ModExp(ulong a, ulong b, ulong N)
			{
				if (b < 0)
				{
					throw new Exception("Exponent must be positive.");
				}
				if (a == 0 && b == 0)
				{
					throw new Exception("0^0 not defined.");
				}
				
				
				/*
				* In order to make sure we dont get overflows, we multiply by a and do the mod at each step.
				* This ensures that we never have result be larger than a*(N-1) (result is never more than
				* N-1 after each for loop, so it could be on a*(N-1). In this project we dont work with N > 10^10
				* so we can work with a up to about 10^5 which is more than enough for our purpouses, we wont really
				* go above bases in double digits).
				*/
				
				/*
				* We dont just do a*a* ... * a, b times, since for b large (which we will need since we exponent to the prime)
				* this takes ages. Instead heres a better way:
				* Write b in binary, and calculate a, then a^2, then a^4 by a^2 * a^2, up to the largest binary digit in b.
				* Multiply all together where we have it. E.g. if we want to find a^42, 42 is 101010 in binary so we
				* do a^32 * a^8 * a^2.
				*/
				
				ulong result = 1;
				
				ulong newA = a;
				ulong newB = b;
				newA = Mod(newA, N);
				
				while (newB > 0)
				{
					if (newB % 2 == 1)
					result = ModMult(result, newA, N);
					
					newB /= 2;
					newA = ModMult(newA, newA, N);
					
					
				}
				//Console.WriteLine(a + "^" + b + " = " + result + " mod " + N);
				return result;
			}
			
			/// <summary>
			/// Works out a*b mod N.
			/// </summary>
			public static int ModMult(int a, int b, int N)
			{
				int result = 0;
				a = Mod(a, N); //initialise a to be in the interval
				
				
				while (b > 0)
				{
					if (b % 2 == 1)
					result = Mod(result + a, N);
					
					a = Mod(a * 2, N);
					b /= 2; //divide b by 2
				}
				
				return result;
			}
			
			/// <summary>
			/// Works out a*b mod N.
			/// </summary>
			public static ulong ModMult(ulong a, ulong b, ulong N)
			{
				ulong result = 0;
				a = Mod(a, N); //initialise a to be in the interval
				
				
				while (b > 0)
				{
					if (b % 2 == 1)
					result = Mod(result + a, N);
					
					a = Mod(a * 2, N);
					b /= 2; //divide b by 2
				}
				
				return result;
			}
			
			/// <summary>
			/// Calculates a mod N. We dont just use a % N since C# is weird with negative numbers. Probably wont use
			/// negative numbers ever in this project but just being careful.
			/// </summary>
			
			public static int Mod(this int a, int N)
			{
				int result = a;
				while (result < 0)
				{
					result += N;
				}
				
				result = result % N;
				return result;
			}
			
			/// <summary>
			/// Calculates a mod N. We dont just use a % N since C# is weird with negative numbers.
			/// </summary>
			
			public static ulong Mod(this ulong a, ulong N)
			{
				ulong result = a;
				while (result < 0)
				{
					result += N;
				}
				
				result = result % N;
				return result;
			}
			
			public static long Mod(this long a, ulong N)
			{
				long result = a;
				while (result < 0)
				{
					result += (long)N;
				}
				
				result = result % (long)N;
				return result;
			}
			
			#endregion
			
			/// <summary>
			/// Calculates a mod N. We dont just use a % N since C# is weird with negative numbers. Make it within -N/2 and N/2.
			/// </summary>
			public static long ModWithinHalf(this ulong a, long N)
			{
				long result = (long)a.Mod((ulong)N);
				
				if (result >= N/2)
				{
					result -= N;
				}
				
				return result;
			}
			
		}
	}
	
\end{lstlisting}

\subsection{PellsEquation.cs}

\begin{lstlisting}
	using System;
	using System.Collections.Generic;
	using System.Linq;
	using System.Text;
	using System.Threading.Tasks;
	
	namespace ContinuedFractionProject
	{
		internal class PellsEquation
		{
			/// <summary>
			/// Does this tuple x, y, N satisfy x^2 - Ny^2 = 1
			/// </summary>
			public static bool SatisfiesPellsEquation(ulong x, ulong y, ulong N)
			{
				return (SatisfiesPMPellsEquation(x, y, N) == (true, 1));
			}
			
			/// <summary>
			/// Does this tuple x, y, N satisfy x^2 - Ny^2 = -1
			/// </summary>
			public static bool SatisfiesNegativePellsEquation(ulong x, ulong y, ulong N)
			{
				return (SatisfiesPMPellsEquation(x, y, N) == (true, -1));
			}
			
			/// <summary>
			/// Does this tuple x, y, N satisfy x^2 - Ny^2 = +-1? Returns which one it satisfies.
			/// </summary>
			public static (bool, int) SatisfiesPMPellsEquation(ulong x, ulong y, ulong N)
			{
				
				int checkPrimesUpTo = 151; //a magic number, as explained in the writeup checking up to here proves the result.
				
				int[] primesToTest = PrimesAndFactorisation.PrimesLessThan(checkPrimesUpTo + 1);
				
				int pmOne = 0; //whether its plus or minus one. needs to be the same for all of 3, ..., 151 (mod 2, 1 = -1)
				
				foreach (int p in primesToTest)
				{
					ulong xSquaredModP = ModularArithmetic.ModMult(x, x, Convert.ToUInt64(p)); //x^2 mod p
					
					ulong ySquaredModP = ModularArithmetic.ModMult(y, y, Convert.ToUInt64(p)); //y^2 mod p
					
					ulong NySquaredModP = ModularArithmetic.ModMult(ySquaredModP, N, Convert.ToUInt64(p)); //Ny^2 mod p
					
					ulong negativeNySquaredModP = Convert.ToUInt64(p) - NySquaredModP; //-Ny^2 mod p, but get it as a positive number as ulong only deals with that
					
					ulong result = ModularArithmetic.Mod(xSquaredModP + negativeNySquaredModP, Convert.ToUInt64(p)); //x^2 - Ny^2 mod p, within the range 0, ..., p-1
					
					
					if (!(result == 1 || result == Convert.ToUInt64(p) - 1)) //its congruent to neither 1 or -1 mod p
					{
						return (false, 0);
					}
					else
					{
						if (p > 2)
						{
							if (p == 3) //first prime we test where 1 != -1, so assign pmOne here
							{
								if (result == 1)
								{
									pmOne = 1;
								}
								else //result is p-1
								{
									pmOne = -1;
								}
							}
							else
							{
								int thispmOne;
								if (result == 1)
								{
									thispmOne = 1;
								}
								else //result is p-1
								{
									thispmOne = -1;
								}
								if (pmOne != thispmOne) //is 1 on some primes > 2, -1 on the others, so is not a solution
								{
									return (false, 0);
								}
							}
						}
						
					}
					
				}
				//passed all the tests so its true, return true and whichever it is
				
				
				
				return (true, pmOne);
			}
			
			/// <summary>
			/// Find a solution to x^2 - Ny^2 = +-1 for a given N using the continued fraction of sqrt(N).
			/// </summary>
			/// <returns>A tuple, x is the first entry and y the second entry. (0, 0) if none found</returns>
			public static (ulong, ulong) FindPMPellSolutionUsingContinuedFractions(ulong N, int checkUpToConvergent, int plusOrMinusOne)
			{
				if (plusOrMinusOne != 1 && plusOrMinusOne != -1)
				{
					throw new Exception("Invalid input, need to provide plus or minus one.");
				}
				//Theorem says that the first solution will be either x_{k-1}, y_{k-1} or x_{2k-1}, y_{2k-1} for k the period of the CFE.
				//project instructions does not necessarily need that so we take a somewhat more naive approach and just check all of them
				
				
				(ulong, ulong)[] unused = new (ulong, ulong)[0];
				ulong[] aValues = ContinuedFractions.ContinuedFractionExpansionOfSquareRoot(N, out unused, checkUpToConvergent);
				
				//Console.WriteLine(aValues.ToFormattedString());
				
				(ulong, ulong)[] PQValues = ContinuedFractions.PartialQuotients(aValues);
				
				foreach ((ulong, ulong) tuple in PQValues)
				{
					ulong x = tuple.Item1;
					ulong y = tuple.Item2;
					
					(bool, int) satisfies = SatisfiesPMPellsEquation(x, y, Convert.ToUInt64(N));
					
					
					
					if (satisfies.Item1 && satisfies.Item2 == plusOrMinusOne) //satisfies relevant pells
					{
						return (x, y);
					}
				}
				
				return (0, 0); //none found
			}
			
			/// <summary>
			/// Find a solution to x^2 - Ny^2 = 1 for a given N using the continued fraction of sqrt(N).
			/// </summary>
			/// <returns>A tuple, x is the first entry and y the second entry. (0, 0) if none found</returns>
			public static (ulong, ulong) FindPellSolutionUsingContinuedFractions(ulong N, int checkUpToConvergent)
			{
				return FindPMPellSolutionUsingContinuedFractions(N, checkUpToConvergent, 1);
			}
			
			/// <summary>
			/// Find a solution to x^2 - Ny^2 = -1 for a given N using the continued fraction of sqrt(N).
			/// </summary>
			/// <returns>A tuple, x is the first entry and y the second entry. (0, 0) if none found</returns>
			public static (ulong, ulong) FindNegativePellSolutionUsingContinuedFractions(ulong N, int checkUpToConvergent)
			{
				return FindPMPellSolutionUsingContinuedFractions(N, checkUpToConvergent, -1);
			}
		}
	}
	
\end{lstlisting}

\subsection{PrimeFieldInverses.cs}

\begin{lstlisting}
	using System;
	using System.Collections.Generic;
	using System.Text;
	
	namespace PrimeFieldInverses
	{
		public static class Inverses
		{
			
			
			
			public static int InverseOf(int integer, int prime)
			{
				if (integer <= 0 || integer >= prime)
				{
					throw new Exception("Integer given is not in the range 1, ..., p-1");
				}
				else
				{
					for (int b = 1; b <= prime - 1; b++)
					{
						int ab = (integer * b) % prime; //the value of ab mod p
						//steps += 2;
						
						if (ab == 1) //b is as inverse.
						{
							return b;
						}
					}
					throw new Exception("Could not find an inverse");
				}
			}
		}
	}
	
\end{lstlisting}

\subsection{PrimesAndFactorisation.cs}

\begin{lstlisting}
	using System;
	using System.Collections.Generic;
	using System.Linq;
	using System.Text;
	using System.Threading.Tasks;
	
	namespace ContinuedFractionProject
	{
		public static class PrimesAndFactorisation
		{
			/// <summary>
			/// Returns whether the given integer is prime.
			/// </summary>
			/// <param name="n"></param>
			/// <returns></returns>
			public static bool IsPrime(this long n)
			{
				if (n <= 0)
				{
					throw new Exception("Need to give a positive integer as input for IsPrime");
				}
				if (n == 1)
				{
					return false;
				}
				for (long i = 2; i <= Math.Sqrt(n); i++) //trial division will be good enough for us in this project
				{
					if (n % i == 0) //composite
					{
						return false;
					}
				}
				return true;
			}
			
			/// <summary>
			/// Returns whether the given integer is prime.
			/// </summary>
			/// <param name="n"></param>
			/// <returns></returns>
			public static bool IsPrime(this int n)
			{
				return ((long)n).IsPrime();
			}
			
			public static bool IsPrime(this ulong n)
			{
				return ((long)n).IsPrime();
			}
			
			/// <summary>
			/// Returns the primes less than a given integer
			/// </summary>
			/// <param name="n"></param>
			/// <returns></returns>
			public static int[] PrimesLessThan(int n)
			{
				if (n <= 1)
				{
					return new int[0];
				}
				List<int> list = new List<int>();
				for (int i = 2; i < n; i++)
				{
					if (i.IsPrime())
					{
						list.Add(i);
					}
				}
				return list.ToArray();
			}
			
			/// <summary>
			/// Can n be written as a product of some given primes, possibly with -1?
			/// </summary>
			/// <param name="n"></param>
			/// <param name="list"></param>
			/// <returns></returns>
			public static bool IsProductOfBNumbers(this long n, int[] list, bool checkIfListIsValid, out (int, int)[] primeFactorisation)
			{
				primeFactorisation = new (int, int)[0];
				
				if (n == 0) //by convention
				{
					return true;
				}
				
				List<(int, int)> primeFactorisationList = new List<(int, int)>();
				if (n < 0 && list.Contains(-1)) //possible if we can write the positive -n as all the things other than -1.
				{
					bool isPositiveBNumberProduct = IsProductOfBNumbers(-n, list.RemoveAll(-1), checkIfListIsValid, out primeFactorisation);
					primeFactorisation.Append((-1, 1)); //factor of -1
					return isPositiveBNumberProduct;
				}
				else if (n > 0 && list.Contains(-1)) //no need for -1
				{
					bool isPositiveBNumberProduct = IsProductOfBNumbers(n, list.RemoveAll(-1), checkIfListIsValid, out primeFactorisation);
					return isPositiveBNumberProduct;
				}
				else if (n < 0) //n is negative, with no -1 in it, so we cant do it.
				{
					return false;
				}
				
				if (checkIfListIsValid)
				{
					foreach (int number in list)
					{
						if (number != -1 && number.IsPrime() == false)
						{
							throw new Exception("Given list contains the composite number " + number);
						}
					}
				}
				
				//valid list or didnt check
				
				//now check if n is a product of the B numbers
				//we know all the things in the list are positive primes at this point due to earlier handled bits
				
				long currentN = n;
				
				foreach (int number in list)
				{
					int timesDividedBy = 0;
					while (currentN % number == 0)
					{
						timesDividedBy++;
						currentN /= number;
					}
					if (timesDividedBy > 0) //has this as a factor at least once
					{
						primeFactorisationList.Add((number, timesDividedBy));
					}
					
				}
				
				if (currentN == 1) //we did it
				{
					primeFactorisation = primeFactorisationList.ToArray();
					return true;
				}
				else //there must be some other stuff not in B.
				{
					return false;
				}
				
			}
			
			public static bool IsProductOfBNumbers(this long n, int[] list, bool checkIfListIsValid)
			{
				(int, int)[] unusedList = new (int, int)[0];
				
				
				return IsProductOfBNumbers(n, list, checkIfListIsValid, out unusedList);
			}
			
			/// <summary>
			/// Returns whether the given integer is a square number.
			/// </summary>
			/// <param name="n"></param>
			/// <returns></returns>
			public static bool IsSquare(this int n)
			{
				return (Math.Sqrt(n) == (int)Math.Sqrt(n));
			}
			
			/// <summary>
			/// Returns whether the given integer is a square number.
			/// </summary>
			/// <param name="n"></param>
			/// <returns></returns>
			public static bool IsSquare(this ulong n)
			{
				return (Math.Sqrt(n) == (ulong)Math.Sqrt(n));
			}
		}
	}
	
\end{lstlisting}

\subsection{Program.cs}

\begin{lstlisting}
	
	using ContinuedFractionProject;
	using System.IO;
	using Matrices;
	using Matrices.Kernel;
	using Matrices.RREF;
	using System;
	using System.Linq;
	
	//Where will we output the results of the program?
	string logFilePath = "";
	string outputFolder = @"C:\Users\robin\Documents\University\Computational Projects\II\15.10 The Continued Fraction Method For Factorization\Report\Program Output\";
	
	
	
	
	Question1();
	Question2();
	Question3();
	Question5();
	Question6();
	Question7();
	
	void Question1()
	{
		
		logFilePath = outputFolder + "Question1.txt";
		string tableFilePath = outputFolder + "Question1Table.txt";
		ClearLogFile();
		ClearFile(tableFilePath);
		
		Log("Running question 1");
		Log("");
		
		
		int[] primesBelow50 = PrimesAndFactorisation.PrimesLessThan(50);
		
		long[] numbersToTestSpecifically = new long[] { 1, 318, 28577, 65536 };
		
		Log("Testing if numbers are B-numbers with B = " + primesBelow50.ToFormattedString());
		Log("n\tB-number\tFactorisation if a B-number");
		foreach (long number in numbersToTestSpecifically)
		{
			(int, int)[] factorisation;
			
			bool BNumber = number.IsProductOfBNumbers(primesBelow50, false, out factorisation);
			Log(number + "\t" + BNumber + "\t" + factorisation.ToFormattedString());
		}
		
		int numberOfSamples = 100000;
		
		Log("");
		Log("Generating max(" + numberOfSamples + ", 10^d - 10^{d-1} + 1) samples of d digit numbers and finding the probability they can be written as a product of B = " + primesBelow50.ToFormattedString());
		Log("");
		
		string tableHeading = "d\tsamples\tB_numbers_found\tprobability";
		Log(tableHeading);
		LogGivenFile(tableHeading, tableFilePath);
		
		for (int d = 1; d <= 14; d++)
		{
			long[] sample = GenericExtensions.DistinctRandomIntegersWithGivenDigits(d, numberOfSamples);
			int actualNumberOfSamples = sample.Length;
			int BNumberCount = 0;
			
			foreach (long number in sample)
			{
				if (number.IsProductOfBNumbers(primesBelow50, false))
				{
					BNumberCount++;
				}
			}
			double probability = (double)BNumberCount / (double)actualNumberOfSamples;
			
			
			string line = d + "\t" + actualNumberOfSamples + "\t" + BNumberCount + "\t" + probability.ToString("0.00000000");
			Log(line);
			LogGivenFile(line, tableFilePath);
		}
		
		
	}
	
	void Question2()
	{
		logFilePath = outputFolder + "Question2.txt";
		string tableFilePath;
		string heading;
		ClearLogFile();
		
		int length = 50;
		
		ulong[] NsToFindCFE = new ulong[] { 7, 13, 19 };
		foreach (ulong N in NsToFindCFE)
		{
			tableFilePath = outputFolder + "Question2Table_" + N + ".txt";
			ClearFile(tableFilePath);
			Log("Calculating the CFE of sqrt(" + N + ")");
			(ulong, ulong)[] RSValues;
			ulong[] AValues = N.ContinuedFractionExpansionOfSquareRoot(out RSValues, length);
			
			Log("A values: " + AValues.ToFormattedString());
			
			(ulong, ulong)[] PQValues = AValues.PartialQuotients();
			
			heading = "n\tr_n\ts_n\ta_n\tp_n\tq_n";
			Log(heading);
			LogGivenFile(heading, tableFilePath);
			for (int i = 0; i < length; i++)
			{
				string tableLine = (i+1) + "\t" + RSValues[i].Item1 + "\t" + RSValues[i].Item2 + "\t" + AValues[i] + "\t" + PQValues[i].Item1 + "\t" + PQValues[i].Item2;
				Log(tableLine);
				LogGivenFile(tableLine, tableFilePath);
			}
		}
		
		//investigate how large r and s can become
		
		tableFilePath = outputFolder + "Question2Table_investigation.txt";
		ClearFile(tableFilePath);
		
		int investigateUpTo = 1000;
		
		int investigateUpToLength = 1000;
		
		//how far should we check maxmaxSOverRootN up to?
		ulong investigateUpToBestMaxMaxSOverRootN = 1000000;
		
		
		Log("Investigating how large r and s become in the first " + investigateUpToLength + " convergents for sqrt(N) up to " + investigateUpTo);
		
		
		heading = "N\tmax_r_found\tmax_s_found\tmax_s_found_over_root_N";
		
		Log(heading);
		LogGivenFile(heading, tableFilePath);
		
		List<ulong> MaxSIsOne = new List<ulong>();
		
		double maxmaxSOverRootN = 0;
		
		for (ulong N = 2; N <= Convert.ToUInt64(investigateUpTo); N++)
		{
			if (N.IsSquare() == false)
			{
				(ulong, ulong)[] RSValues;
				N.ContinuedFractionExpansionOfSquareRoot(out RSValues, investigateUpToLength);
				//remark: the values for a will have overflows here, but we dont care about them, r and s are sufficiently small (as defined by recursion relations) that we dont care
				
				ulong maxR = RSValues.Select(i => i.Item1).Max();
				ulong maxS = RSValues.Select(i => i.Item2).Max();
				
				double maxSOverRootN = (double)maxS / Math.Sqrt(N);
				
				string line = N + "\t" + maxR + "\t" + maxS + "\t" + maxSOverRootN;
				
				if (maxS == 1)
				{
					MaxSIsOne.Add(N);
				}
				if (maxSOverRootN > maxmaxSOverRootN)
				{
					maxmaxSOverRootN = maxSOverRootN;
				}
				Log(line);
				LogGivenFile(line, tableFilePath);
			}
			
			
		}
		
		
		
		Log("max s is one at " + MaxSIsOne.ToArray().ToFormattedString());
		
		
		for (ulong N = 2; N <= investigateUpToBestMaxMaxSOverRootN; N++)
		{
			if (N.IsSquare() == false)
			{
				(ulong, ulong)[] RSValues;
				N.ContinuedFractionExpansionOfSquareRoot(out RSValues, investigateUpToLength);
				//remark: the values for a will have overflows here, but we dont care about them, r and s are sufficiently small (as defined by recursion relations) that we dont care
				
				ulong maxS = RSValues.Select(i => i.Item2).Max();
				
				double maxSOverRootN = (double)maxS / Math.Sqrt(N);
				
				
				if (maxSOverRootN > maxmaxSOverRootN)
				{
					maxmaxSOverRootN = maxSOverRootN;
				}
			}
			
			if (N % 100000 == 0)
			{
				Log("maximum of max s over root N found up to " + N + " is " + maxmaxSOverRootN);
			}
		}
		
		
	}
	
	void Question3()
	{
		logFilePath = outputFolder + "Question3.txt";
		string tableFilePath = outputFolder + "Question3Table.txt";
		ClearLogFile();
		ClearFile(tableFilePath);
		
		
		Log("Testing select (x, y, N) pairs to see if they solve Pell's equation.");
		
		
		(ulong, ulong, ulong)[] xyNPairs = new (ulong, ulong, ulong)[] { (32080051, 3115890, 106), (8890182, 851525, 109), (253293,35989,113) };
		
		
		foreach ((ulong, ulong, ulong) pair in xyNPairs)
		{
			
			(bool, int) satisfies = PellsEquation.SatisfiesPMPellsEquation(pair.Item1, pair.Item2, pair.Item3);
			
			if (satisfies.Item1)
			{
				Console.WriteLine("(x,y,N) = (" + pair.Item1 + ", " + pair.Item2 + ", " + pair.Item3 + ") satisfies x^2 - Ny^2 = " + satisfies.Item2);
			}
			else
			{
				Console.WriteLine("(x,y,N) = (" + pair.Item1 + ", " + pair.Item2 + ", " + pair.Item3 + ") does not satisfy x^2 - Ny^2 = +-1");
			}
			
			
		}
		
		
		Log("Solutions to x^2 - Ny^2 = 1 and x'^2 - Ny'^2 = -1 for required N:");
		string heading = "N\tx\ty\tx'\ty'";
		
		Log(heading);
		LogGivenFile(heading, tableFilePath);
		
		for (ulong N = 1; N <= 100; N++)
		{
			Question3PellTest(N, tableFilePath);
		}
		for (ulong N = 500; N <= 550; N++)
		{
			Question3PellTest(N, tableFilePath);
		}
	}
	
	void Question3PellTest(ulong N, string tableFilePath)
	{
		//Theorem says that the first solution will be either x_{k-1}, y_{k-1} or x_{2k-1}, y_{2k-1} for k the period of the CFE.
		//project instructions does not necessarily need that so we take a somewhat more naive approach and just check all partial convergents
		//up to some number. Let's check it up to 2k-1 then.
		
		int period = ContinuedFractions.PeriodOfContinuedFractionExpansionOfSquareRoot(N);
		
		(ulong, ulong) pellSol = PellsEquation.FindPellSolutionUsingContinuedFractions(N, 2 * period);
		(ulong, ulong) negativePellSol = PellsEquation.FindNegativePellSolutionUsingContinuedFractions(N, period - 1);
		
		
		string line = N.ToString();
		
		
		if (pellSol == (0, 0)) //we failed to find one!
		{
			//throw new Exception("Did not find a solution for " + N + ".");
			line += "\t\t";
			
		}
		else
		{
			line += "\t" + pellSol.Item1 + "\t" + pellSol.Item2;
		}
		if (negativePellSol == (0, 0)) //we failed to find one!
		{
			//throw new Exception("Did not find a solution for " + N + ".");
			line += "\t\t";
			
		}
		else
		{
			line += "\t" + negativePellSol.Item1 + "\t" + negativePellSol.Item2;
		}
		
		Log(line);
		LogGivenFile(line, tableFilePath);
	}
	
	void Question5()
	{
		logFilePath = outputFolder + "Question5.txt";
		string tableFilePath = outputFolder + "Question5Table.txt";
		ClearLogFile();
		ClearFile(tableFilePath);
		
		ulong[] Nvalues = new ulong[] { 2012449237, 2575992413, 3548710699 };
		int upTo = 50;
		
		
		
		
		List<(ulong, ulong)[]> results = new List<(ulong, ulong)[]>();
		
		string heading = "n";
		
		foreach (ulong N in Nvalues)
		{
			results.Add(ContinuedFractions.ConvergentNumeratorsModulo(N, upTo+1)); //find the results
			
			heading += "\tP_n_mod_" + N + "\t" + "P_n^2_mod_" + N;
		}
		
		Log(heading);
		LogGivenFile(heading, tableFilePath);
		
		
		for (int i = 0; i <= upTo; i++)
		{
			string line = i + "\t";
			for (int j = 0; j < Nvalues.Length; j++)
			{
				line += "\t" + results[j][i].Item1 + "\t" + results[j][i].Item2;
			}
			Log(line);
			LogGivenFile(line, tableFilePath);
		}
		
		int upToFindMatches = 500;
		Log("Matches:");
		//See if we have matches.
		foreach (ulong N in Nvalues)
		{
			Log("\tFor N = " + N + ":");
			(ulong, ulong)[] values = ContinuedFractions.ConvergentNumeratorsModulo(N, upToFindMatches + 1);
			
			ulong[] squares = values.Select(x => x.Item2).ToArray();
			
			ulong[] duplicatedSquares = squares.AllEntriesThatAppearMoreThanOnce();
			
			foreach (ulong duplicatedSquare in duplicatedSquares)
			{
				Log("\t\tThe square " + duplicatedSquare + " is duplicated at:");
				int[] indices = squares.IndicesOf(duplicatedSquare);
				
				Log("\t\tindex\tP_n mod N\tP_n^2 mod N");
				
				foreach (int index in indices)
				{
					Log("\t\t" + index + "\t" + values[index].Item1 + "\t" + values[index].Item2 );
				}
			}
			
		}
		
	}
	
	//Gaussian elimination of a matrix
	void Question6()
	{
		logFilePath = outputFolder + "Question6.txt";
		//string tableFilePath = outputFolder + "Question5Table.txt";
		ClearLogFile();
		//ClearFile(tableFilePath);
		(int, int)[] sizesToGenerate = new (int, int)[] { (2, 2), (2, 2), (2, 3), (3, 2), (3, 3), (3, 3), (3, 4), (4, 3), (2, 4), (4,2) };
		
		foreach ((int, int) sizeToGenerate in sizesToGenerate)
		{
			
			Matrix m;
			
			do
			{
				m = Matrix.RandomMatrix(2, sizeToGenerate.Item1, sizeToGenerate.Item2);
			} while (m.AllEntriesZero());
			
			Matrix rref = m.GetReducedRowEchelonForm(2);
			
			
			Matrices.Vector[] basis = rref.GetBasisOfKernel(2);
			
			string basisToString;
			
			if (basis.Length == 0)
			{
				basisToString = "\\emptyset";
			}
			else
			{
				
				
				basisToString = "\\left\\{";
					
					foreach (Matrices.Vector v in basis)
					{
						basisToString += "" + v.ToLaTeX() + ", ";
					}
					basisToString = basisToString.Substring(0, basisToString.Length - 2);
					
					basisToString += "\\right\\}";
			}
			
			
			
			
			Log("$" + m.ToLaTeX() + "$ & $" + rref.ToLaTeX() + "$ & $" + basisToString + "$\\\\");
		}
		
		
		
	}
	
	void Question7()
	{
		logFilePath = outputFolder + "Question7.txt";
		//
		ClearLogFile();
		
		
		//Part one of question 7, test some N
		
		
		
		
		ulong[] numbersToTest = new ulong[] { 2012449237, 2575992413, 3548710699, 377691131, 175224311, 48958009, 483205427 };
		
		Log("Testing the following N: " + numbersToTest.ToFormattedString());
		
		foreach (ulong number in numbersToTest)
		{
			
			int convergentsRequired;
			long foundFactor = Question7WithNumber(number, out convergentsRequired, true, 0, 50);
			
			if (foundFactor == 0)
			{
				Log(number + " is prime");
				
				Log("");
			}
			else
			{
				Log("Factored " + number + " as " + foundFactor + " x " + number / (ulong)foundFactor);
				
				Log("");
			}
			
			Log("Was able to figure out the factorisation of " + number + " in " + convergentsRequired + " convergents.");
			
			
		}
		
		Log("");
		
		
		
		//part 2 of question 7, Investigate the number of convergents typically required for factorization.
		
		string tableFilePath = outputFolder + "Question7ConvergentInvestigationTable.txt";
		
		ClearFile(tableFilePath);
		
		int giveUpAfter = 1500;
		
		for (ulong i = 1; i <= 100; i++) //was: 1000000
		{
			int numberOfConvergentsRequired;
			
			
			
			Question7WithNumber(i, out numberOfConvergentsRequired, false, giveUpAfter, 50);
			
			if (numberOfConvergentsRequired == -1) //failed
			{
				//Log(i + "\t>" + giveUpAfter);
				Console.WriteLine(i + "\t>" + giveUpAfter);
				LogGivenFile(i + "\t>" + giveUpAfter, tableFilePath);
			}
			else
			{
				//Log(i + "\t" + numberOfConvergentsRequired);
				Console.WriteLine(i + "\t" + numberOfConvergentsRequired);
				LogGivenFile(i + "\t" + numberOfConvergentsRequired, tableFilePath);
			}
		}
		
		Log("[Rest of testing omitted in output.]");
		
		Log("");
		
		Log("Third part of question 7, investigate different choices of B");
		
		giveUpAfter = 10000;
		
		tableFilePath = outputFolder + "Question7BChoices.txt";
		
		ClearFile(tableFilePath);
		
		//part 3 of question 7, investigate different choices of B
		
		string header = "";
		header += "primeUpTo\t";
		
		
		foreach (ulong number in numbersToTest)
		{
			header += number.ToString();
			header += "\t";
		}
		
		Log(header);
		LogGivenFile(header, tableFilePath);
		
		
		for (int numberOfPrimes = 10; numberOfPrimes <= 200; numberOfPrimes += 10)
		{
			
			string line = "";
			
			line += numberOfPrimes + "\t";
			foreach (ulong number in numbersToTest)
			{
				int convergentsRequired;
				Question7WithNumber(number, out convergentsRequired, false, giveUpAfter, numberOfPrimes);
				line += convergentsRequired + "\t";
			}
			Log(line);
			LogGivenFile(line, tableFilePath);
		}
		
	}
	
	
	long Question7WithNumber(ulong N, out int convergentsRequired, bool printOutDetails, int giveAfterNumberOfConvergents, int primeUpTo)
	{
		convergentsRequired = 0;
		if (printOutDetails)
		{
			Log("Running Q7 with N=" + N);
		}
		
		
		if (N.IsSquare()) //special case
		{
			return (long)Math.Sqrt(N);
		}
		else if (N.IsPrime()) //special case
		{
			return 0;
		}
		
		int[] B = PrimesAndFactorisation.PrimesLessThan(primeUpTo);
		
		
		
		B = B.Append(-1).ToArray();
		
		if (printOutDetails)
		{
			Log("B = " + B.ToFormattedString());
			
		}
		
		
		//B is the set as described in the project
		
		int upToConvergent = 50;
		
		
		
		bool foundFactorisation = false;
		
		do
		{
			if (upToConvergent > giveAfterNumberOfConvergents && giveAfterNumberOfConvergents > 0)
			{
				convergentsRequired = -1;
				return 0;
			}
			if (printOutDetails)
			{
				Log("Going to check up to n=" + upToConvergent + " for the convergents. Will increase if nothing is found.");
			}
			
			
			//First, we find some B numbers
			
			
			
			(ulong, ulong)[] RSValues;
			
			//get the p_n mod N and p_n^2 mod N
			(ulong, ulong)[] convergents = ContinuedFractions.ConvergentNumeratorsModulo(N, upToConvergent);
			
			//The way we do it, p_n^2 mod N is within 0 and N-1. Want it within -N/2, N/2.
			(ulong, long)[] adjustedConvergents = convergents.Select(x => (x.Item1, x.Item2.ModWithinHalf((long)N))).ToArray();
			
			
			
			//filter out which ones are B numbers in their p_n^2 entry. also keep track of n
			
			List<(ulong, long, int)> adjustedConvergentsBNumbers = new List<(ulong, long, int)>();
			
			int n = 0;
			
			if (printOutDetails)
			{
				Log("We found the following B numbers in the continued fraction of sqrt(N)");
				
				Log("n\tp_n mod N\t<p_n^2>'_N");
			}
			
			
			
			foreach ((ulong, long) pnpnsquaredpair in adjustedConvergents)
			{
				
				if (pnpnsquaredpair.Item2.IsProductOfBNumbers(B, false)) //is it a B number?
				{
					
					adjustedConvergentsBNumbers.Add((pnpnsquaredpair.Item1, pnpnsquaredpair.Item2, n));
					
					if (printOutDetails)
					{
						Log(n + "\t" + pnpnsquaredpair.Item1 + "\t" + pnpnsquaredpair.Item2);
					}
					
					
					
					
				}
				n++;
			}
			
			if (adjustedConvergentsBNumbers.Count > 0)
			{
				//we now have the B numbers. make the matrix for which we will use as outlined in Q7 in the project.
				
				
				int numberOfBNumbers = adjustedConvergentsBNumbers.Count;
				
				int sizeOfB = B.Length;
				
				Matrix primeParityMatrix = new Matrix(sizeOfB, numberOfBNumbers);
				
				int column = 1;
				foreach ((ulong, long, int) pairWithbNumber in adjustedConvergentsBNumbers.ToArray())
				{
					long bNumber = pairWithbNumber.Item2;
					
					int row = 1;
					foreach (int primeOrMinusOne in B)
					{
						int parity = 0;
						//what is the parity of primeOrMinusOne in the factorisation of the b number?
						
						if (primeOrMinusOne == -1) //do this separately
						{
							if (bNumber < 0)
							{
								parity = 1;
							}
						}
						else
						{
							long current = bNumber;
							while (current % primeOrMinusOne == 0)
							{
								parity++;
								current /= primeOrMinusOne;
							}
						}
						
						parity = parity % 2;
						
						primeParityMatrix[row, column] = parity;
						
						
						row++;
					}
					
					column++;
				}
				
				
				//We have constructed the prime parity matrix.
				
				if (printOutDetails)
				{
					Log("Prime parity matrix:");
					
					Log(primeParityMatrix.ToLaTeX());
				}
				
				
				
				//find its kernel. sort it by minimal number of convergents required.
				
				Vector[] kernel = primeParityMatrix.GetBasisOfKernel(2);
				
				
				kernel = kernel.OrderBy(x => x.LastNonZeroIndex()).ToArray();
				
				if (printOutDetails)
				{
					Log("Found " + kernel.Length + " vectors in the kernel. Sorting them through and checking them by latest index:");
					
					foreach (Vector vector in kernel)
					{
						Log("\t" + vector.ToString());
					}
				}
				
				
				
				
				if (kernel.Length > 0) //we found a valid vector.
				{
					
					if (printOutDetails)
					{
						Log("Found non-trivial vectors in the prime parity matrix kernel which we will now iterate through.");
					}
					
					//loop through vectors in the kernel until we find a valid factorisation.
					
					foreach (Vector kernelVector in kernel)
					{
						
						if (printOutDetails)
						{
							Log("Testing if " + kernelVector.ToString() + " gives a valid factorisation.");
						}
						
						
						
						//we try to find x, y such that x^2 = y^2 like this
						
						long x = 1; //x
						
						int[] elementsOfBFactorCountInY = new int[B.Length];
						
						
						for (int i = 1; i <= kernelVector.Size; i++)
						{
							if (kernelVector[i] == 1)
							{
								x *= (long)adjustedConvergentsBNumbers[i - 1].Item1;
								x = x.Mod(N);
								
								long pnSquaredMod = adjustedConvergentsBNumbers[i - 1].Item2;
								
								for (int j = 0; j < B.Length; j++)
								{
									int elementOfB = B[j];
									
									if (elementOfB != -1) //ignore -1, its irrelevant when taing a square root as we know we have an even no of them
									{
										long current = pnSquaredMod;
										
										while (current % elementOfB == 0)
										{
											current /= elementOfB;
											elementsOfBFactorCountInY[j]++;
										}
									}
								}
								
								//productOfPnSquared *= adjustedConvergentsBNumbers[i - 1].Item2;
								// productOfPnSquared = productOfPnSquared.Mod(N);
							}
						}
						
						long y = 1;
						
						for (int j = 0; j < B.Length; j++)
						{
							int numberOfTimes = elementsOfBFactorCountInY[j]; //we know this must be even by the squareness required
							
							int number = B[j];
							
							for (int i = 0; i < numberOfTimes / 2; i++)
							{
								y *= (long)number;
								y = y.Mod(N);
							}
						}
						
						//we have y^2, we want to find y
						
						if (printOutDetails)
						{
							Log("Found the pair x=" + x + ", y=" + y);
						}
						
						
						
						//Hopefully now, x =/= +- y mod N
						
						if (x.Mod(N) != y.Mod(N) && x.Mod(N) != y.Mod(N))
						{
							//hopefully now, (N, x-y) and (N, x+y) are non trivial factors
							
							long nonTrivialFactorCandidate1 = GenericExtensions.GCD((long)N, (x - y).Mod(N));
							
							long nonTrivialFactorCandidate2 = GenericExtensions.GCD((long)N, (x - y).Mod(N));
							
							//bool works = false;
							
							
							
							if (nonTrivialFactorCandidate1 != 1 && nonTrivialFactorCandidate1 != (long)N)
							{
								if ((long)N % nonTrivialFactorCandidate1 == 0)
								{
									//works = true;
									
									if (printOutDetails)
									{
										Log("Found the factor (N,x+y)=" + nonTrivialFactorCandidate1);
									}
									
									//worked! how many convergents did we need?
									
									convergentsRequired = adjustedConvergentsBNumbers[kernelVector.LastNonZeroIndex()-1].Item3;
									
									return nonTrivialFactorCandidate1;
								}
							}
							else if (nonTrivialFactorCandidate2 != 1 && nonTrivialFactorCandidate2 != (long)N)
							{
								if ((long)N % nonTrivialFactorCandidate2 == 0)
								{
									//works = true;
									
									if (printOutDetails)
									{
										Log("Found the factor (N, x-y)=" + nonTrivialFactorCandidate2);
									}
									
									//worked! how many convergents did we need?
									
									convergentsRequired = adjustedConvergentsBNumbers[kernelVector.LastNonZeroIndex()].Item3;
									
									return nonTrivialFactorCandidate2;
								}
							}
							else
							{
								
								if (printOutDetails)
								{
									Log("No non-trivial factors found. (N, x+y)=" + nonTrivialFactorCandidate1 + ", (N, x-y)=" + nonTrivialFactorCandidate2);
								}
								
							}
						}
						
						else
						{
							if (printOutDetails)
							{
								Log("Has x = +- y");
							}
							
						}
					}
					
					
				}
				
				else
				{
					//did not find something
					
					if (printOutDetails)
					{
						Log("Couldn't find a valid vector. Searching further.");
					}
					
				}
			}
			
			
			
			upToConvergent += 50;
		} while (true);
		
		
		
		
		
		
		
	}
	
	
	void Log(string str)
	{
		Console.WriteLine(str);
		LogGivenFile(str, logFilePath);
		
	}
	void LogGivenFile(string str, string path)
	{
		
		
		if (!File.Exists(path))
		{
			File.Create(path).Close();
		}
		using (StreamWriter sw = File.AppendText(path))
		{
			sw.WriteLine(str);
		}
		
	}
	
	void ClearLogFile()
	{
		ClearFile(logFilePath);
	}
	
	void ClearFile(string path)
	{
		File.Delete(path);
		File.Create(path).Close();
	}
\end{lstlisting}
				
\newpage
\section{Appendix B: Output}
\label{appendix_b_output}
				
\subsection{Question 1 output}
\label{output_question_1}
%\lstinputlisting{Program Output/Question1.txt}

\end{document}
